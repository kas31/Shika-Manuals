\section{NECTA Physics Exam Format}
\label{sec:physics-format}

%%%%%%%%%%%%%%%%%%%%%%%%%%%%%
%-----------------------------Form II------------------------------------
%%%%%%%%%%%%%%%%%%%%%%%%%%%%%
\subsection{Form II}
\noindent The following format for the Form Two National Assessment (FTNA) is based on the revised version of the Physics Syllabus for Ordinary Secondary Education of 2007. The exam is intended to assess the competences acquired by the students after two years of study. The following information was taken from The National Examinations Council of Tanzania - Form Two National Assessment Formats %CITE!!!!!!!!!!!!!!!!!!

\subsubsection{General Objectives}
\noindent The general objectives of the Physics assessment are to test students' ability to:
\begin{enumerate}[topsep=1ex,itemsep=0ex,partopsep=1ex,parsep=1ex]
	\item Demonstrate laboratory practice and safety
	\item Develop skills on basic principles of scientific investigation
	\item Develop skills for making physical measurements
	\item Recognize behavior and properties of matter
	\item Understand concepts and principles of magnetism and electricity
	\item Comprehend the laws of motion
	\item Understand principles of simple machines
	\item Develop knowledge on sustainable energy for environmental conservation
\end{enumerate}

\subsubsection{General Competences}
\noindent The FTNA will specifically test the students' ability to:
\begin{enumerate}[topsep=1ex,itemsep=0ex,partopsep=1ex,parsep=1ex]
	\item Practice safety rules in daily life
	\item Apply basic principles of scientific investigation
	\item Make appropriate measurements of physical quantities
	\item Use scientific skills to identify nature and properties of matter
	\item Apply electricity and magnetism knowledge in daily life
	\item Apply laws of motion in dealing with moving objects
	\item Use simple machines to simplify work
	\item Practice environmental conservation by adopting appropriate sustainable energy sources
\end{enumerate}

\subsubsection{Assessment Rubric}
\noindent The assessment consists of \textbf{one (1)} theory paper. The duration of the exam is \textbf{2 hours and 30 minutes}. The paper consists of \textbf{ten (10)} questions categorized into sections A, B, and C. The students are required to attempt \textbf{all} questions from \textbf{all} sections. 
\begin{enumerate}
	\item \textbf{Section A} \\
	This section has \textbf{three (3)} objective questions. This section weighs a total of \textbf{thirty (30)} marks.
	\begin{enumerate}
		\item Question 1 is composed of 20 multiple choice items derived from various topics (20 marks)
		\item Question 2 is composed of 5 matching items (5 marks)
		\item Question 3 is composed of 5 filling in the blank items derived from various topics (5 marks)
	\end{enumerate}
	
	\item \textbf{Section B} \\
	This section has \textbf{five (5)} short answer questions (10 marks each). This section weighs a total of \textbf{fifty (50)} marks.
	
	\item \textbf{Section C} \\
	This section has \textbf{two (2)} questions (10 marks each). The questions are aimed at assessing students' knowledge and skills in drawing and management of Physics apparati and simple technological devices in everyday life. This section weighs a total of \textbf{twenty (20)} marks. 
\end{enumerate}

\subsubsection{Assessment Topics}
\noindent The following table shows the topics assessed in the FTNA as well as the percentage they have appeared in past examinations. 
\begin{center}
	\begin{tabular}{|c|l|c|c|} \hline
		Form & \multicolumn{1}{|c|}{Topic} & Percentage [\%] & Total Percentage [\%] \\ \hline
		\multirow{9}{*}{I} 	& Introduction to Physics 					& 1.25 & \multirow{9}{*}{47.76}	\\ \cline{2-3}
						& Introduction to Laboratory Practice 		& 0.97 & \\ \cline{2-3}
						& Measurements						& 10.58 & \\ \cline{2-3}
						& Forces 								& 4.32 & \\ \cline{2-3}
						& Archimedes Principle and Law of Floatation 	& 3.90 & \\ \cline{2-3}
						& Structure and Properties of Matter 			& 7.52 & \\ \cline{2-3}
						& Pressure							& 4.32 & \\ \cline{2-3}
						& Work, Energy and Power				& 6.82 & \\ \cline{2-3}
						& Light 								& 8.08 & \\ \hline
		\multirow{9}{*}{II} 	& Static Electricity	 					& 4.87 & \multirow{9}{*}{43.02} \\ \cline{2-3}
						& Current Electricity				 		& 8.08 & \\ \cline{2-3}
						& Magentism							& 5.15 & \\ \cline{2-3}
						& Forces in Equilibrium					& 3.90 & \\ \cline{2-3}
						& Simple Machines					 	& 7.24 & \\ \cline{2-3}
						& Motion in Straight Lines		 			& 5.01 & \\ \cline{2-3}
						& Newton's Laws of Motion				& 3.76 & \\ \cline{2-3}
						& Temperature							& 3.34 & \\ \cline{2-3}
						& Sustainable Energy Sources				& 1.67 & \\ \hline
	\end{tabular}
\end{center}

%%%%%%%%%%%%%%%%%%%%%%%%%%%%%
%-----------------------------Form IV------------------------------------
%%%%%%%%%%%%%%%%%%%%%%%%%%%%%
\subsection{Form IV}
\noindent The following format for the Form Four Certificate of Secondary Education (CSEE) is based on the revised version of the Physics Syllabus for Ordinary Secondary Education of 2007. The exam is intended to assess the competences acquired by the students after four years of study. The following information was taken from The National Examinations Council of Tanzania - Certificate of Secondary Education Examination Formats %CITE!!!!!!!!!!!!!!!!!!

\subsubsection{General Objectives}
\noindent The general objectives of the Physics examination are to test students' ability to:
\begin{enumerate}[topsep=1ex,itemsep=0ex,partopsep=1ex,parsep=1ex]
	\item Develop knowledge of concepts, laws, theories and principles of Physics
	\item Use procedures of scientific investigation
	\item Use scientific principles on conservation and suitable use of the environment
	\item Demonstrate manipulative skills to manage various technological appliances
	\item Develop the language of communication in physics
\end{enumerate}

\subsubsection{General Competences}
\noindent The CSEE will specifically test the students' ability to:
\begin{enumerate}[topsep=1ex,itemsep=0ex,partopsep=1ex,parsep=1ex]
	\item Use Physics knowledge, principles and concepts in daily life
	\item Demonstrate scientific methods in solving problems in daily life
	\item Demonstrate technological skills in conservation and sustainable use of the environment
	\item Manage simple technological appliances
	\item Use the language of Physics in communication
\end{enumerate}

\subsubsection{Examination Rubric}
\noindent The examination consists of \textbf{two (2)} papers. The first paper focuses on theory while the second focuses on practicals. This section will discuss the format for the Physics theory paper. To see the format of the Physics practical paper, please refer to Section \ref{phy-practical-format}. \\

\noindent The duration of the theory exam is \textbf{3 hours}. The paper consists of \textbf{eleven (11)} questions categorized into sections A, B, and C. The students are required to attempt \textbf{all} questions from sections A and B and \textbf{one} question from section C. 
\begin{enumerate}
	\item \textbf{Section A} \\
	This section has \textbf{three (3)} objective questions. This section weighs a total of \textbf{thirty (30)} marks.
	\begin{enumerate}
		\item Question 1 is composed of 10 multiple choice items derived from various topics (10 marks)
		\item Question 2 is composed of 10 matching items (10 marks)
		\item Question 3 is composed of 10 filling in the blank items derived from various topics (10 marks)
	\end{enumerate}
	
	\item \textbf{Section B} \\
	This section has \textbf{six (6)} long answer questions (10 marks each). This section weighs a total of \textbf{sixty (60)} marks.
	
	\item \textbf{Section C} \\
	This section has \textbf{two (2)} questions (10 marks each). The questions are aimed at assessing students' knowledge and skills in drawing and management of Physics apparati and simple technological devices in everyday life. This section weighs a total of \textbf{ten (10)} marks. 
\end{enumerate}


\subsubsection{Examination Topics}
\noindent The following table shows the topics assessed in the CSEE as well as the percentage they have appeared in past examinations. 
\begin{center}
	\begin{longtable}{|c|l|c|c|} 
	
		\hline Form & \multicolumn{1}{|c|}{Topic} & Percentage [\%] & Total Percentage [\%] \\ \hline
		\endfirsthead
		
		\hline Form & \multicolumn{1}{|c|}{Topic} & Percentage [\%] & Total Percentage [\%] \\ \hline
		\endhead
		
		\multirow{9}{*}{I} 	& Introduction to Physics 					& 0.00 & \multirow{9}{*}{14.14} 	\\ \cline{2-3}
						& Introduction to Laboratory Practice 		& 0.00 & 					\\ \cline{2-3}
						& Measurements						& 3.08 & 					\\ \cline{2-3}
						& Forces 								& 1.03 & 					\\ \cline{2-3}
						& Archimedes Principle and Law of Floatation 	& 1.94 & 					\\ \cline{2-3}
						& Structure and Properties of Matter 			& 1.60 & 					\\ \cline{2-3}
						& Pressure							& 2.05 & 					\\ \cline{2-3}
						& Work, Energy and Power				& 2.74 & 					\\ \cline{2-3}
						& Light 								& 1.71 & 					\\ \hline
		\multirow{9}{*}{II} 	& Static Electricity	 					& 3.42 & \multirow{9}{*}{21.32}	\\ \cline{2-3}
						& Current Electricity				 		& 2.17 & 					\\ \cline{2-3}
						& Magentism							& 2.39 & 					\\ \cline{2-3}
						& Forces in Equilibrium					& 2.85 & 					\\ \cline{2-3}
						& Simple Machines					 	& 1.94 & 					\\ \cline{2-3}
						& Motion in Straight Lines		 			& 3.31 & 					\\ \cline{2-3}
						& Newton's Laws of Motion				& 3.19 & 					\\ \cline{2-3}
						& Temperature							& 1.37 & 					\\ \cline{2-3}
						& Sustainable Energy Sources				& 0.68 & 					\\ \hline
		\multirow{9}{*}{III}	& Application of Vectors					& 0.80 & \multirow{9}{*}{28.96}	\\ \cline{2-3}
						& Friction								& 0.23 & 					\\ \cline{2-3}
						& Light								& 6.73 & 					\\ \cline{2-3}
						& Optical Instruments					& 0.80 & 					\\ \cline{2-3}
						& Thermal Expansion					& 3.76 & 					\\ \cline{2-3}
						& Transfer of Thermal Heat				& 1.14 & 					\\ \cline{2-3}
						& Measurement of Thermal Energy			& 3.31 & 					\\ \cline{2-3}
						& Vapour and Humidity					& 1.25 & 					\\ \cline{2-3}
						& Current Electricity 						& 10.95 & 					\\ \hline
		\multirow{7}{*}{IV}	& Waves								& 6.61 & \multirow{7}{*}{35.58}	\\ \cline{2-3}
						& Electromagnetism						& 4.45 & 					\\ \cline{2-3}
						& Radioactivity							& 8.44 & 					\\ \cline{2-3}
						& Thermionic Emission					& 3.42 & 					\\ \cline{2-3}
						& Electronics							& 6.16 & 					\\ \cline{2-3}
						& Astronomy							& 4.56 & 					\\ \cline{2-3}
						& Geophysics							& 1.94 & 					\\ \hline
	\end{longtable}
\end{center}

