\section{NECTA Chemistry Exam Format}
\label{sec:chemistry-format}

%%%%%%%%%%%%%%%%%%%%%%%%%%%%%
%-----------------------------Form II------------------------------------
%%%%%%%%%%%%%%%%%%%%%%%%%%%%%
\subsection{Form II}
\noindent The following format for the Form Two National Assessment (FTNA) is based on the revised version of the Chemistry Syllabus for Ordinary Secondary Education of 2010. The exam is intended to assess the competences acquired by the students after two years of study. The following information was taken from The National Examinations Council of Tanzania - Form Two National Assessment Formats. %CITE!!!!!!!!!!!!!!!!!!

\subsubsection{General Objectives}
\noindent The general objectives of the Chemistry assessment are to test students' ability to:
\begin{enumerate}
	\item Apply Chemistry knowledge and skills to solve daily life problems
	\item Understand nature and properties of matter
	\item Apply Chemistry knowledge and skills in proper use and management of the environment
	\item Apply Chemistry knowledge and skills in performing various activities in the laboratory
\end{enumerate}

\subsubsection{General Competences}
\noindent The FTNA will specifically test the students' ability to:
\begin{enumerate}
	\item Demonstrating Chemistry knowledge and skills in solving daily life problems
	\item Identifying various Chemistry apparatus and its uses
	\item Applying basic principles of scientific procedure
	\item Using fuels efficiently, treating and purifying water with environment consideration
	\item Explaining regulation and rules guiding proper use of laboratory
	\item Explaining various concepts related to nature and matter
\end{enumerate}

\subsubsection{Assessment Rubric}
\noindent The assessment consists of \textbf{one (1)} theory paper. The duration of the exam is \textbf{2 hours and 30 minutes}. The paper consists of \textbf{ten (10)} questions categorized into sections A and B. The students are required to attempt \textbf{all} questions from \textbf{all} sections. 
\begin{enumerate}
	\item \textbf{Section A} \\
	This section has \textbf{two (2)} objective questions. This section weighs a total of \textbf{twenty (20)} marks.
	\begin{enumerate}
		\item Question 1 is composed of 10 multiple choice items derived from various topics (10 marks)
		\item Question 2 is composed of 10 matching or filling in the blank questions derived from various topics (10 marks)
	\end{enumerate}
	
	\item \textbf{Section B} \\
	This section has \textbf{eight (8)} short answer questions (10 marks each) derived from various topics. This section weighs a total of \textbf{eighty (80)} marks.
\end{enumerate}

\subsubsection{Assessment Topics}
\noindent The following table shows the topics assessed in the FTNA as well as the percentage they have appeared in past examinations. 
\begin{center}
	\begin{tabular}{|c|l|c|} \hline
		Form & \multicolumn{1}{|c|}{Topic} & Percentage [\%] \\ \hline
		\multirow{6}{*}{I} 	& Introduction to Chemistry				& \\ \cline{2-3}
						& Laboratory Techniques and Safety 		& \\ \cline{2-3}
						& Heat Sources and Flames				& \\ \cline{2-3}
						& The Scientific Procedure				& \\ \cline{2-3}
						& Matter							 	& \\ \cline{2-3}
						& Air Combustion, Rusting and Fire Fighting	& \\ \hline
		\multirow{7}{*}{II} 	& Oxygen			 					& \\ \cline{2-3}
						& Hydrogen					 		& \\ \cline{2-3}
						& Water								& \\ \cline{2-3}
						& Fuels and Energy						& \\ \cline{2-3}
						& Atomic Structure					 	& \\ \cline{2-3}
						& Periodic Classification		 			& \\ \cline{2-3}
						& Formula, Bonding and Nomenclature		& \\ \hline
	\end{tabular}
\end{center}

%%%%%%%%%%%%%%%%%%%%%%%%%%%%%
%-----------------------------Form IV------------------------------------
%%%%%%%%%%%%%%%%%%%%%%%%%%%%%
\subsection{Form IV}
\noindent The following format for the Form Four Certificate of Secondary Education (CSEE) is based on the revised version of the Chemistry Syllabus for Ordinary Secondary Education of 2007. The exam is intended to assess the competences acquired by the students after four years of study. The following information was taken from The National Examinations Council of Tanzania - Certificate of Secondary Education Examination Formats. %CITE!!!!!!!!!!!!!!!!!!

\subsubsection{General Objectives}
\noindent The general objectives of the Chemistry examination are to test students' ability to:
\begin{enumerate}
	\item Apply Chemistry knowledge, skills and principles in everyday life activities
	\item Design and perform experiments
	\item Understand symbols, formulae and equations to communicate in Chemistry
	\item Apply the scientific principles and knowledge in exploitation of natural resources with conservation of the environment
\end{enumerate}

\subsubsection{General Competences}
\noindent The CSEE will specifically test the students' ability to:
\begin{enumerate}
	\item Ability to demonstrate Chemistry knowledge, skills and principles in solving daily life problems
	\item Developing knowledge on Chemistry by doing various activities and/or experiments
	\item Ability to demonstrate chemical symbols, formulae and equations to communicate in Chemistry
	\item Using science and technological skills in conserving and making sustainable use of the environment
\end{enumerate}

\subsubsection{Examination Rubric}
\noindent The examination consists of \textbf{two (2)} papers. The first paper focuses on theory while the second focuses on practicals. This section will discuss the format for the Chemistry theory paper. To see the format of the Chemistry practical paper, please refer to Section \ref{chem-practical-format}. \\

\noindent The duration of the theory exam is \textbf{3 hours}. The paper consists of \textbf{thirteen (13)} questions categorized into sections A and B. The students are required to attempt \textbf{all} questions from \textbf{all} sections. 
\begin{enumerate}
	\item \textbf{Section A} \\
	This section has \textbf{two (2)} objective questions. This section weighs a total of \textbf{twenty (20)} marks.
	\begin{enumerate}
		\item Question 1 is composed of 10 multiple choice items derived from various topics (10 marks)
		\item Question 2 is composed of 10 matching questions derived from various topics (10 marks)
	\end{enumerate}
	
	\item \textbf{Section B} \\
	This section has \textbf{nine (9)} short answer questions (6 marks each) derived from various topics. Each question will have \textbf{two (2)} items. This section weighs a total of \textbf{fifty-four (54)} marks.
	
	\item \textbf{Section C} \\
	This section has \textbf{two (2)} essay questions without items derived from various topics. This section weighs a total of \textbf{twenty-six (26)} marks. 

\end{enumerate}
	
	
\subsubsection{Examination Topics}
\noindent The following table shows the topics assessed in the CSEE as well as the percentage they have appeared in past examinations. 
\begin{center}
	\begin{tabular}{|c|l|c|} \hline
		Form & \multicolumn{1}{|c|}{Topic} & Percentage [\%] \\ \hline
		\multirow{6}{*}{I} 	& Introduction to Chemistry				& \\ \cline{2-3}
						& Laboratory Techniques and Safety 		& \\ \cline{2-3}
						& Heat Sources and Flames				& \\ \cline{2-3}
						& The Scientific Procedure				& \\ \cline{2-3}
						& Matter							 	& \\ \cline{2-3}
						& Air Combustion, Rusting and Fire Fighting	& \\ \hline
		\multirow{7}{*}{II} 	& Oxygen			 					& \\ \cline{2-3}
						& Hydrogen					 		& \\ \cline{2-3}
						& Water								& \\ \cline{2-3}
						& Fuels and Energy						& \\ \cline{2-3}
						& Atomic Structure					 	& \\ \cline{2-3}
						& Periodic Classification		 			& \\ \cline{2-3}
						& Formula, Bonding and Nomenclature		& \\ \hline
		\multirow{9}{*}{III}	& Chemical Equations					& \\ \cline{2-3}
						& Hardness of Water 					& \\ \cline{2-3}
						& Acids, Bases and Salts					& \\ \cline{2-3}
						& The Mole Concept and Related Calculations	& \\ \cline{2-3}
						& Volumetric Analysis					& \\ \cline{2-3}
						& Ionic Theory and Electrolysis				& \\ \cline{2-3}
						& Chemical Kinetics, Equilibrium and Energetic & \\ \cline{2-3}
						& Extraction of Metals					& \\ \cline{2-3}
						& Compounds of Metals					& \\ \hline
		\multirow{5}{*}{IV}	& Non-metals and their Compounds 			& \\ \cline{2-3}
						& Organic Chemistry						& \\ \cline{2-3}
						& Soil Chemistry						& \\ \cline{2-3}
						& Pollution							& \\ \cline{2-3}
						& Qualitative Analysis					& \\ \hline
	\end{tabular}
\end{center}



















