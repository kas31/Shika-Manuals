\section{NECTA Biology Exam Format}
\label{sec:biology-format}

%%%%%%%%%%%%%%%%%%%%%%%%%%%%%
%-----------------------------Form II------------------------------------
%%%%%%%%%%%%%%%%%%%%%%%%%%%%%
\subsection{Form II}
\noindent The following format for the Form Two National Assessment (FTNA) is based on the revised version of the Biology Syllabus for Ordinary Secondary Education of 2005. The exam is intended to assess the competences acquired by the students after two years of study. The following information was taken from The National Examinations Council of Tanzania - Form Two National Assessment Formats %CITE!!!!!!!!!!!!!!!!!!

\subsubsection{General Objectives}
\noindent The general objectives of the Biology assessment are to test students' ability to:
\begin{enumerate}[topsep=1ex,itemsep=0ex,partopsep=1ex,parsep=1ex]
	\item Evaluate the role, influence and importance of biological science in everyday life
	\item Demonstrate the capacity to improve and maintain their health, families and the community
	\item Apply scientific skills and procedures in interpreting various biological data
	\item Apply basic biological knowledge and appropriate skills in combating problems related to environment, health disorders and diseases such as HIV/AIDS, STIs, etc. 
	\item Demonstrate necessary biological practical skills
\end{enumerate}

\subsubsection{General Competences}
\noindent The FTNA will specifically test the students' ability to:
\begin{enumerate}[topsep=1ex,itemsep=0ex,partopsep=1ex,parsep=1ex]
	\item Apply scientific procedures and practical skills in studying Biology
	\item Demonstrate the appropriate use of biological knowledge, concepts, principles and skills in everyday life
	\item Demonstrate preventive measures and precautions against common accidents and other related health problems
	\item Apply biological knowledge in combating health related problems such as HIV/AIDS, STIs, malaria, cholera and other communicable diseases
	\item Apply biological knowledge, skills and scientific principles to improve and maintain their own health, the health of the families and the community
	\item Demonstrate biological skills in writing scientific procedures, observations and report writing
	\item Analyze groups of organisms according to their similarities and differences
	\item Preserve nature and ensure sustained interaction of organisms in the natural environment
	\item Evaluate roles of various physiological processes, construct diagrams of biological structures and systems and indicate their functions in plants and animals
\end{enumerate}

\subsubsection{Assessment Rubric}
\noindent The assessment consists of \textbf{one (1)} theory paper. The duration of the exam is \textbf{2 hours and 30 minutes}. The paper consists of \textbf{eleven (11)} questions categorized into sections A, B, and C. The students are required to attempt \textbf{all} questions from sections A and B and \textbf{one} question from section C. 
\begin{enumerate}
	\item \textbf{Section A} \\
	This section has \textbf{four (3)} objective questions. This section weighs a total of \textbf{thirty (30)} marks.
	\begin{enumerate}
		\item Question 1 is composed of 10 multiple choice items derived from various topics (10 marks)
		\item Question 2 is composed of 10 true and false questions derived from various topics (10 marks)
		\item Question 3 is composed of 5 matching items derived from one of the topics (5 marks)
		\item Question 4 is composed of 5 filling in the blanks derived from one of the topics (5 marks)
	\end{enumerate}
	
	\item \textbf{Section B} \\
	This section has \textbf{five (5)} short answer questions (10 marks each). This section weighs a total of \textbf{fifty (50)} marks.
	
	\item \textbf{Section C} \\
	This section has \textbf{two (2)} essay questions (20 marks each). This section weighs a total of \textbf{twenty (20)} marks. 
\end{enumerate}

\subsubsection{Assessment Topics}
\noindent The following table shows the topics assessed in the FTNA as well as the percentage they have appeared in past examinations. 
\begin{center}
	\begin{tabular}{|c|l|c|} \hline
		Form & \multicolumn{1}{|c|}{Topic} & Percentage [\%] \\ \hline
		\multirow{5}{*}{I} 	& Introduction to Biology					& \\ \cline{2-3}
						& Safety in Our Environment				& \\ \cline{2-3}
						& Health and Immunity					& \\ \cline{2-3}
						& Cell Structure and Organization			& \\ \cline{2-3}
						& Classification of Living Things			& \\ \hline
		\multirow{5}{*}{II} 	& Classification of Living Things			& \\ \cline{2-3}
						& Nutrition						 		& \\ \cline{2-3}
						& Balance of Nature						& \\ \cline{2-3}
						& Transport of Materials in Living Things		& \\ \cline{2-3}
						& Gaseous Exchange and Respiration		& \\ \hline
	\end{tabular}
\end{center}

%%%%%%%%%%%%%%%%%%%%%%%%%%%%%
%-----------------------------Form IV------------------------------------
%%%%%%%%%%%%%%%%%%%%%%%%%%%%%
\subsection{Form IV}
\noindent The following format for the Form Four Certificate of Secondary Education (CSEE) is based on the revised version of the Biology Syllabus for Ordinary Secondary Education of 2005. The exam is intended to assess the competences acquired by the students after four years of study. The following information was taken from The National Examinations Council of Tanzania - Certificate of Secondary Education Examination Formats %CITE!!!!!!!!!!!!!!!!!!

\subsubsection{General Objectives}
\noindent The general objectives of the Biology examination are to test students' ability to:
\begin{enumerate}[topsep=1ex,itemsep=0ex,partopsep=1ex,parsep=1ex]
	\item Evaluate the role, influence and importance of biological science in everyday life
	\item Develop the capacity to improve and maintain their own health, of families and the community
	\item Develop mastery of fundamental concepts, principles, and skills of biological science and related fields, such as agriculture, medicine, pharmacy and veterinary
	\item Develop necessary biological practical skills
	\item Ability to demonstrate scientific skills and procedures in interpreting various biological data
	\item Acquire basic knowledge and apply appropriate skills in combating problems related to HIV/AIDS/STIs, gender, population, environment, drug/substance abuse, sexual and reproductive health
	\item Develop the ability and desire for self-study, self-confidence and self-advancement in biological sciences and related fields
\end{enumerate}

\subsubsection{General Competences}
\noindent The CSEE will specifically test the students' ability to:
\begin{enumerate}[topsep=1ex,itemsep=0ex,partopsep=1ex,parsep=1ex]
	\item Make appropriate use of biological knowledge, concepts, skills and principles in solving various problems in daily life
	\item Record, analyze and interpret data from scientific investigations, using appropriate methods and technology to generate relevant information in biological science
	\item Demonstrate knowledge and skills in combating health related problems such as HIV/AIDS, drugs and drug abuse, sexual and reproductive health
	\item Access relevant information on biological science and related fields for self-study and life-long learning
\end{enumerate}

\subsubsection{Examination Rubric}
\noindent The examination consists of \textbf{two (2)} papers. The first paper focuses on theory while the second focuses on practicals. This section will discuss the format for the Biology theory paper. To see the format of the Biology practical paper, please refer to Section \ref{bio-practical-format}. \\

\noindent The duration of the exam is \textbf{3 hours}. The paper consists of \textbf{thirteen (13)} questions categorized into sections A, B, and C. The students are required to attempt \textbf{all} questions from sections A and B and \textbf{one} question from section C. 
\begin{enumerate}
	\item \textbf{Section A} \\
	This section has \textbf{two (2)} objective questions (10 marks each). This section weighs a total of \textbf{twenty (20)} marks.
	
	\item \textbf{Section B} \\
	This section has \textbf{eight (8)} structured short answer questions. Each question will be divided into two parts. This section weighs a total of \textbf{sixty (60)} marks. Mark allocation will be indicated at the end of each question. 
	
	\item \textbf{Section C} \\
	This section has \textbf{three (3)} long answer\slash essay questions (20 marks each). The answer for this question will have to be comprehensive and include as many points as possible. This section weighs a total of \textbf{twenty (20)} marks. 
\end{enumerate}


\subsubsection{Examination Topics}
\noindent The following table shows the topics assessed in the CSEE as well as the percentage they have appeared in past examinations. 
\begin{center}
	\begin{tabular}{|c|l|c|} \hline
		Form & \multicolumn{1}{|c|}{Topic} & Percentage [\%] \\ \hline
		\multirow{5}{*}{I} 	& Introduction to Biology					& \\ \cline{2-3}
						& Safety in Our Environment				& \\ \cline{2-3}
						& Health and Immunity					& \\ \cline{2-3}
						& Cell Structure and Organization			& \\ \cline{2-3}
						& Classification of Living Things			& \\ \hline
		\multirow{5}{*}{II} 	& Classification of Living Things			& \\ \cline{2-3}
						& Nutrition						 		& \\ \cline{2-3}
						& Balance of Nature						& \\ \cline{2-3}
						& Transport of Materials in Living Things		& \\ \cline{2-3}
						& Gaseous Exchange and Respiration		& \\ \hline
		\multirow{6}{*}{III}	& Classification of Living Things 			& \\ \cline{2-3}
						& Movement 							& \\ \cline{2-3}
						& Coordination 							& \\ \cline{2-3}
						& Excretion							& \\ \cline{2-3}
						& Regulation							& \\ \cline{2-3}
						& Reproduction							& \\ \hline
		\multirow{5}{*}{IV}	& Growth								& \\ \cline{2-3}
						& Genetics 							& \\ \cline{2-3}
						& Classification of Living Things			& \\ \cline{2-3}
						& Evolution							& \\ \cline{2-3}
						& HIV/AIDs and STIs						& \\ \hline
	\end{tabular}
\end{center}
