\subsection{The Mole Concept and Related Calculations}

\begin{enumerate}
	\item A student attempted to prepare hydrogen gas by reacting zinc metal with dilute sulphuric acid. In this experiment zinc metal granules of about 0.5 cm diameter and 0.20 moles of acid were used. The rate of formation of hydrogen gas was found to be slow. 
	\begin{enumerate}[topsep=0ex,itemsep=0ex,partopsep=1ex,parsep=1ex]
		\item[i)] Explain three ways in which the rate of formation of hydrogen gas could be increased.
		\item[ii)] If the student wanted 36 cm$^3$ of hydrogen gas at s.t.p., what amount of the acid would be required?
	\end{enumerate}
	
	\item How many moles of oxygen are required for the complete combustion of 2.2 g of C$_3$H$_8$ to form carbon dioxide and water?
		\begin{enumerate}[topsep=0ex,itemsep=0ex,partopsep=1ex,parsep=1ex]
		\item[(A)] 0.050 moles
		\item[(B)] 0.15 moles
		\item[(C)] 0.25 moles
		\item[(D)] 0.50 moles
		\item[(E)] 0.025 moles
	\end{enumerate}
	
	\item Compound X contains 24.24\% carbon, 4.04\% hydrogen and 71.72\% chlorine. Given that, the vapour density of X is 49.5.
		\begin{enumerate}[topsep=0ex,itemsep=0ex,partopsep=1ex,parsep=1ex]
		\item[i)] Calculate the molecular formula of the compound X
		\item[ii)] Draw and name the displayed\slash open structure formula of the possible isomer(s) from the molecular formula determined.
	\end{enumerate}
	
\end{enumerate}









