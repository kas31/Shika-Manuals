\subsection{Extraction of Metals}

\begin{enumerate}
	\item State four steps employed in the extraction of moderate reactive metals
	
	\item Copper can be obtained from the ore, copper pyrites (CuFeS$_2$). The ore is heated in a limited amount of air giving the following reaction:
	\begin{center}
		4CuFeS$_2$ + 11O$_2$ $\rightarrow$ 4Cu + 2Fe$_2$O$_3$ + 8SO$_2$
	\end{center}
	\begin{enumerate}[topsep=0ex,itemsep=0ex,partopsep=1ex,parsep=1ex]
		\item[i)] Calculate the maximum mass of copper that can be obtained from 367 kg of copper pyrites
		\item[ii)] State why the gaseous product from this reaction must not be allowed to escape into the atmosphere
	\end{enumerate}
	
	\item How long must a current of 4.00 A be applied to a solution of Cu$^{2+}$ (aq) to produce 2.0 grams of copper metal?
	\begin{enumerate}[topsep=0ex,itemsep=0ex,partopsep=1ex,parsep=1ex]
		\item[(A)] 2.4 $\times$ 10$^4$ s
		\item[(B)] 1.5 $\times$ 10$^3$ s
		\item[(C)] 7.6 $\times$ 10$^2$ s
		\item[(D)] 3.8 $\times$ 10$^2$ s
		\item[(E)] 12 $\times$ 10$^4$ s
	\end{enumerate}
	
	\item State three properties that make aluminium useful in overhead cables
	
	\item 
	\begin{enumerate}[topsep=0ex,itemsep=0ex,partopsep=1ex,parsep=1ex]
		\item[i)] Explain, in terms of electronic configurations, why sodium and potassium elements have similar chemical properties.
		\item[ii)] State the trend in reactivity of group I elements in the Periodic Table and give reasons for it.
	\end{enumerate}
	
	\item Describe the extraction of iron from the haematite ore and write all the chemical equations for the reactions involved in each stage of extraction.

\end{enumerate}








