\subsection{Statistics}

\begin{enumerate}
	\item 
	\begin{enumerate}[topsep=0ex,itemsep=0ex,partopsep=1ex,parsep=1ex]
		\item[(a)] The number of motorcycle accidents which were recorded in one region in Tanzania for seven weeks during November and December 2013 were 14, 2, 12, 4, 10, 6, and 8. Find
		\begin{enumerate}[topsep=0ex,itemsep=0ex,partopsep=1ex,parsep=1ex]
			\item[i)] The mean number of accidents
			\item[ii)] The variance of the accidents
		\end{enumerate}
		
		\item[(b)] The table below shows the height of avocado trees in an orchard
		\begin{center}
			\begin{tabular}{|l|c|c|c|c|c|c|} \hline
				Height ($\times 10^{-1}$ m) & 2 - 6 & 7 - 11 & 12 - 16 & 17 - 21 & 22 - 26 & 27 - 31 \\ \hline
				Frequency & 12 & 14 & 18 & 15 & 4 & 8 \\ \hline
			\end{tabular}
		\end{center}
		\begin{enumerate}[topsep=0ex,itemsep=0ex,partopsep=1ex,parsep=1ex]
			\item[i)] Use the data to draw the histogram
			\item[ii)] Estimate the mode from the histogram in (b) i) above. 
		\end{enumerate}
	\end{enumerate}

	\item
	\begin{enumerate}[topsep=0ex,itemsep=0ex,partopsep=1ex,parsep=1ex]
		\item[(a)] Define the following terms as they are used in statistics:
		\begin{enumerate}[topsep=0ex,itemsep=0ex,partopsep=1ex,parsep=1ex]
			\item[i)] Range
			\item[ii)] Class size
		\end{enumerate}
		
		\item[(b)] The manager of Gold Mining Company recorded the number of absent workers in 52 working days as shown in the table below:
		\begin{center}
			\begin{tabular}{|l|c|c|c|c|c|} \hline
				Number of absent workers & 5 - 9 & 10 - 14 & 15 - 19 & 20 - 24 & 25 - 29 \\ \hline
				Frequency & 6 & 9 & 18 & 16 & 3 \\ \hline
			\end{tabular}
		\end{center}
		Use this data to construct the cumulative frequency curve.
		
		\item[(c)] The following data shows time in seconds which was recorded by a teacher in a swimming competition of students from Precious Beach High School.
		\begin{center}
			\begin{tabular}{ccccccccccc}
				32 & 31 & 27 & 30 & 29 & 27 & 25 & 29 & 26 & 26 & 32 \\
				32 & 25 & 31 & 31 & 27 & 24 & 26 & 26 & 32 & 33 & 28 \\
				26 & 33 & 24 & 28 & 32 & 29 & 32 & 24 & 31 & 27 & 30 \\
				31 & 25 & 29 & 25 & 27 & 30 & 26 \\
			\end{tabular}
		\end{center}
		\begin{enumerate}[topsep=0ex,itemsep=0ex,partopsep=1ex,parsep=1ex]
			\item[i)] Prepare the frequency distribution using the class intervals of 0-4, 5-9, etc. 
			\item[ii)] Determine the standard deviation. 
		\end{enumerate}
	\end{enumerate}
	
	\item The following were the scores obtained by 22 students from Sarawak Secondary School in a mathematics classroom test: 
	\begin{center}
		49, 64, 38, 60, 46, 64, 68, 42, 28, 68, 57, 63, 57, 63, 76, 51, 54, 66, 62, 63, 58, 59, 47, 55
	\end{center}
	\begin{enumerate}[topsep=0ex,itemsep=0ex,partopsep=1ex,parsep=1ex]
		\item[(a)] Summarize the scores in a frequency table with equal class intervals of size 5. Take the lowest limit to be 35. 
		\item[(b)] Find the mean score by using the data in part (a)
		\item[(c)] Find the interquartile range
		\item[(d)] How many students scored above the mean score?
	\end{enumerate}
	
\end{enumerate}












