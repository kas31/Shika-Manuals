\subsection{Atomic Structure}

\begin{enumerate}
	\item Define the following term: Element
	
	\item The mass number of an atom is determined by:
	\begin{enumerate}[topsep=0ex,itemsep=0ex,partopsep=1ex,parsep=1ex]
		\item[(A)] Protons and neutrons
		\item[(B)] Protons and electrons
		\item[(C)] Neutrons and electrons
		\item[(D)] Protons alone
	\end{enumerate}
	
	\item Fill in the blank: The arrangement of electrons in different shells in the atom is called \rule{1.5cm}{0.15mm}.
	
%%%%%%%%%%%%%%%%%%%%%%%%%%%%%%%%%%%%%%%%%%%%%%%%%%%%%%%
	\item An atom M has an atomic number 14 and mass number 28.
	\begin{enumerate}
		\item[i)] What is the number of protons and neutrons?
		\item[ii)] Write the electronic configuration of atom M
	\end{enumerate}

	\item The mass number of a carbon atom that contains six protons, eight neutrons, and six electrons is:
	\begin{enumerate}[topsep=0ex,itemsep=0ex,partopsep=1ex,parsep=1ex]
		\item[(A)] 6
		\item[(B)] 14
		\item[(C)] 8
		\item[(D)] 12
		\item[(E)] 20
	\end{enumerate}
	
	\item Protons, neutrons and electrons particles are located in the atoms; fill in the missing information in below table about these particles. 
	\begin{center}
		\begin{tabular}{|c|c|c|c|} \hline
			\textbf{Particles} & \textbf{Relative Mass} & \textbf{Relative Charge} & \textbf{Location} \\ \hline
			Proton & & & \\ \hline
			Electron & $\frac{1}{1840}$ & & \\ \hline
			Neutron & & 0 & In the nucleus \\ \hline
		\end{tabular}
	\end{center}
	
\end{enumerate}








