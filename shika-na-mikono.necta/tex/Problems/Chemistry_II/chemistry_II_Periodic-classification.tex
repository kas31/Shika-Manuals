\subsection{Periodic Classification}

\begin{enumerate}
	\item An element X with atomic number 16 belongs to:
	\begin{enumerate}[topsep=0ex,itemsep=0ex,partopsep=1ex,parsep=1ex]
		\item[(A)] period 3, group III, valency of 2
		\item[(B)] period 3, group VI, valency of 2
		\item[(C)] period 3, group VI, valency of 6
		\item[(D)] period 6, group VI, valency of 6
	\end{enumerate}
	
	\item Which of the following electron configurations are of metals?
	\begin{enumerate}[topsep=0ex,itemsep=0ex,partopsep=1ex,parsep=1ex]
		\item[(A)] 2:8:1 and 2:5
		\item[(B)] 2:8:2 and 2:6
		\item[(C)] 2:8:3 and 2:8:8:1
		\item[(D)] 2:8:6 and 2:8:8:7
	\end{enumerate}
	
	\item Which of the following is a metal?
	\begin{enumerate}[topsep=0ex,itemsep=0ex,partopsep=1ex,parsep=1ex]
		\item[(A)] Water
		\item[(B)] Chlorine
		\item[(C)] Sodium
		\item[(D)] Nitrogen
	\end{enumerate}
	
	\item Which neutral atom has the same number of electrons as Mg$^{2+}$?
	\begin{enumerate}[topsep=0ex,itemsep=0ex,partopsep=1ex,parsep=1ex]
		\item[(A)] Magnesium
		\item[(B)] Sodium
		\item[(C)] Neon
		\item[(D)] Argon
	\end{enumerate}
	
%%%%%%%%%%%%%%%%%%%%%%%%%%%%%%%%%%%%%%%%%%%%%%%%%%
	\item Which of the following is the electronic configuration of an element Y found in period 3 and group II of the periodic table?
	\begin{enumerate}[topsep=0ex,itemsep=0ex,partopsep=1ex,parsep=1ex]
		\item[(A)] 2:8
		\item[(B)] 2:8:2
		\item[(C)] 2:6
		\item[(D)] 2:8:8:2
		\item[(E)] 2:8:4
	\end{enumerate}

	\item Use the knowledge of the periodic table to complete the table below. 
	\begin{center}
		\begin{tabular}{|cp{9.5cm}|cp{2cm}|} \hline
			\multicolumn{2}{|c|}{List A} & \multicolumn{2}{|c|}{List B} \\ \hline
			(i) & Its hydroxide is used in soil treatment & A & Barium \\ 
			(ii) & It is obtained from its ore in the blast furnace & B & Lithium \\ 
			(iii) & It gives a lilac colour when placed in a non-luminous flame & C & Iron \\
			(iv) & It forms an insoluble sulphate & D & Potassium \\
			(v) & It is in the same group in the periodic table with nitrogen & E & Oxygen \\
			(vi) & It reacts with hydrogen to form a compound which is a liquid & F & Fluorine \\
			& at room temperature & G & Sulphur \\
			(vii) & It is used in filament lamps & H & Argon \\
			(viii) & It is the strongest oxidizing agent among the halogens & I & Phosphorus \\
			(ix) & It exists in three main forms & J & Sodium \\
			(x) & Its chloride is added to food in order to give taste & K & Magnesium \\
			& & L & Carbon \\
			& & M & Neon \\
			& & N & Silicon \\
			& & O & Calcium \\ \hline
		\end{tabular}
	\end{center}
	
\end{enumerate}












