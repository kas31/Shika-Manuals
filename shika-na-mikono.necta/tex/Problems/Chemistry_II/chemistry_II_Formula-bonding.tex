\subsection{Formula Bonding and Nomenclature}

\begin{enumerate}
	\item The simplest formulas of a compound formed when combining 13g of aluminium and 17g of chlorine is
	\begin{enumerate}[topsep=0ex,itemsep=0ex,partopsep=1ex,parsep=1ex]
		\item[(A)] AlCl
		\item[(B)] Al$_2$Cl
		\item[(C)] Al$_3$Cl$_2$
		\item[(D)] AlCl$_3$
	\end{enumerate}
	
	\item A certain compound K contains 15.8\% carbon and 84.2\% sulphur. The molar mass of K is 76 g/mol. Determine its:
	\begin{enumerate}[topsep=0ex,itemsep=0ex,partopsep=1ex,parsep=1ex]
		\item[i)] simplest formula
		\item[ii)] molecular formula
	\end{enumerate}	
	
	\item Write the chemical formula for each of the following compounds:
	\begin{itemize}[topsep=0ex,itemsep=0ex,partopsep=1ex,parsep=1ex]
		\item[i)] Sodium carbonate
		\item[ii)] Calcium nitrate
		\item[iii)] Ammonium chloride
	\end{itemize}
	
	\item State the valency of the following atoms: Aluminium, Neon, Sulphur, Potassium
	
	\item Give the chemical formula for the combination of the following sets of ions: Mg$^{2+}$, PO$_4^{3+}$; Fe$^{3+}$, SO$_4^{2-}$
	
	\item Find the oxidation number of each of the underlined elements in the following: K\underline{Cl}O$_3$; \underline{Cr$_2$}O$_7^{2-}$
	
	\item Use the IUPAC system to name each of the following chemical compounds: CuO, CaSO$_4$, HNO$_3$, ZnCl$_2$
	
	\item Ability of an atom to gain or attract electrons towards itself \rule{1.5cm}{0.15mm}
	
	\item Write the names of the following radicals: SO$_3^{2-}$; ClO$_3^-$; PO$_4^{3-}$
	
	\item Calculate the oxidation state of the underlined element in the following compounds: NH$_4$\underline{Cl}; Al$_2$\underline{O$_3$}; Na$_2$\underline{S}O$_4$; \underline{Ca}(NO$_3$)$_2$

	\item The following table shows the name and the chemical formula of the product formed when ions combine together. Complete filling the table.
	\begin{center}
		\begin{tabular}{|p{3cm}|p{3cm}|p{3cm}|p{3cm}|} \hline
			Ion & Ion & Name & Formula \\ \hline
			Ca$^{2+}$ & Cl$^-$ & Calcium chloride & \\ \hline
			Al$^{3+}$ & SO$_4^{2-}$ & & Al$_2$(SO$_4$)$_3$ \\ \hline
			H$^+$ & & Hydrogen sulphate & \\ \hline
		\end{tabular}
	\end{center}
	
%%%%%%%%%%%%%%%%%%%%%%%%%%%%%%%%%%%%%%%%%%%%%%%%%%%%%%%%%%%%%	
	\item Chlorine ion, Cl$^-$ differs from chlorine atom because it has:
	\begin{enumerate}[topsep=0ex,itemsep=0ex,partopsep=1ex,parsep=1ex]
		\item[(A)] More protons
		\item[(B)] Less protons
		\item[(C)] More electrons
		\item[(D)] Less electrons
		\item[(E)] More neutrons
	\end{enumerate}
	
	\item 
	\begin{enumerate}[topsep=0ex,itemsep=0ex,partopsep=1ex,parsep=1ex]
		\item[i)] What type of a chemical bond is found between fluorine atoms in a fluorine molecule?
		\item[ii)] Name other type(s) of chemical bond formed by fluorine with other elements. Give an example of a compound in which fluorine form this type of bond.
	\end{enumerate}
	
\end{enumerate}













