\subsection{Nonmetals and their Compounds}

\begin{enumerate}
	\item Which of the following sets of elements is arranged in order of increasing electronegativity?
	\begin{enumerate}[topsep=0ex,itemsep=0ex,partopsep=1ex,parsep=1ex]
		\item[(A)] Chlorine, fluorine, nitrogen, oxygen, carbon
		\item[(B)] Fluorine, chlorine, oxygen, nitrogen, carbon
		\item[(C)] Carbon, nitrogen, oxygen, chlorine, fluorine
		\item[(D)] Nitrogen, oxygen, carbon, fluorine, chlorine
		\item[(E)] Fluorine, nitrogen, oxygen, chlorine, carbon
	\end{enumerate}
	
	\item Write balanced chemical equations to show how chlorine reacts with the following:
	\begin{enumerate}[topsep=0ex,itemsep=0ex,partopsep=1ex,parsep=1ex]
		\item[i)] water
		\item[ii)] aqueous iron (II) chloride solution
		\item[iii)] hydrogen sulphide
	\end{enumerate}
	
	\item Write the chemical formula of tetrachloromethane and state the type of bond that exists.
	
	\item The preparation of ammonia in the laboratory is done by heating any ammonium salt with an alkali.
	\begin{enumerate}[topsep=0ex,itemsep=0ex,partopsep=1ex,parsep=1ex]
		\item[i)] Write a balanced chemical equation for the preparation of ammonia gas.
		\item[ii)] State two uses of ammonia.
	\end{enumerate}
	
	\item Which among the following equations correctly shows the reaction between chlorine gas and water?
	\begin{enumerate}[topsep=0ex,itemsep=0ex,partopsep=1ex,parsep=1ex]
		\item[(A)] Cl$_{2 (g)}$ + H$_2$O$_{(l)}$ $\rightarrow$ Cl$_{2 (g)}$
		\item[(B)] 2Cl$_{2 (g)}^-$ + 2H$_2$O$_{(l)}$ $\rightarrow$ 4Cl$_{(aq)}$ + O$_{2 (g)}$ + 2H$_{2 (g)}$
		\item[(C)] Cl$_{2 (g)}$ + H$_2$O$_{(l)}$ $\rightarrow$ HCl$_{(aq)}$ + HOCl$_{(aq)}$
		\item[(D)] 2Cl$_{2 (g)}$ + 2H$_2$O$_{(l)}$ $\rightarrow$ 2HOCl$_{2 (aq)}$ + H$_{2 (aq)}$
		\item[(E)] 2Cl$_{2 (g)}$ + 3H$_2$O$_{(l)}$ $\rightarrow$ Cl$_{2 (g)}$ + 2H$_3$O$^+$
	\end{enumerate}
	
	\item Which among the following pair of substances are allotropes?
	\begin{enumerate}[topsep=0ex,itemsep=0ex,partopsep=1ex,parsep=1ex]
		\item[(A)] H$_2$O and H$_2$O$_2$
		\item[(B)] $^12$C and $^14$C
		\item[(C)] P$_4$ and P$_8$
		\item[(D)] H$_2$ and 2H$^+$
		\item[(E)] H$^+$ and H$_3$O
	\end{enumerate}
	
	\item Briefly explain what will happen when 
	\begin{enumerate}[topsep=0ex,itemsep=0ex,partopsep=1ex,parsep=1ex]
		\item[i)] concentrated sulphuric acid is exposed to the atomosphere.
	\end{enumerate}

\end{enumerate}







