\section{NECTA Mathematics Exam Format}
\label{sec:math-format}

%%%%%%%%%%%%%%%%%%%%%%%%%%%%%
%-----------------------------Form II------------------------------------
%%%%%%%%%%%%%%%%%%%%%%%%%%%%%
\subsection{Form II}
\noindent The following format for the Form Two National Assessment (FTNA) is based on the revised version of the Basic Mathematics Syllabus for Ordinary Secondary Education of 2005. The exam is intended to assess the competences acquired by the students after two years of study. The following information was taken from The National Examinations Council of Tanzania - Form Two National Assessment Formats %CITE!!!!!!!!!!!!!!!!!!

\subsubsection{General Objectives}
\noindent The general objectives of the Basic Mathematics assessment are to test students' ability to:
\begin{enumerate}[topsep=1ex,itemsep=0ex,partopsep=1ex,parsep=1ex]
	\item Perform computations on numbers, algebraic terms and radicals
	\item Use approximations in solving simple problems
	\item Convert and do computations on basic units, decimals, percentages and fractions
	\item Construct and draw geometrical figures as well as finding angles, perimeters and areas of simple geometrical figures
	\item Compute ratios, profit and loss
	\item Draw graphs of linear equations, solve linear equations in one or two unknowns, solve linear inequalities in one unknown, solve quadratic equations and transpose formulae
	\item Derive and apply the laws of exponents and logarithms
	\item Do calculations using mathematical tables
	\item Prove and apply congruence and similarity of figures
	\item Represent reflections, rotations, translations and enlargement geometrically
	\item Determine sine, cosine and tangent of angles and hence apply them in solving problems
	\item Represent and interpret statistical data collected from real life situations
	\item Perform operations on sets and apply sets to solve problems
\end{enumerate}

\subsubsection{General Competences}
\noindent The FTNA will specifically test the students' ability to:
\begin{enumerate}[topsep=1ex,itemsep=0ex,partopsep=1ex,parsep=1ex]
	\item Distinguish different types of numbers
	\item Estimate and compute numbers accurately
	\item Convert units, decimals, percentages and fractions
	\item Handle mathematical instruments in constructing and drawing geometrical figures
	\item Solve problems on geometry, ratio, profit and loss
	\item Draw graph and interpret linear equations
	\item Find perimeters and areas of simple geometrical figures
	\item Find relationships among logarithms, exponents, radicals, right angled triangles and trigonometric ratios
	\item Use mathematical tables in computations
	\item Verify laws and prove theorems
	\item Do scale drawing and geometrical transformations
	\item Solve problems on quadratic equations
	\item Organize and interpret data
	\item Apply set operations in solving problems
\end{enumerate}

\subsubsection{Assessment Rubric}
\noindent The assessment consists of \textbf{one (1)} theory paper. The duration of the exam is \textbf{2 hours and 30 minutes}. The paper consists of \textbf{ten (10)} short-response questions (10 marks each). The students are required to attempt \textbf{all} questions, showing all work clearly.  

\subsubsection{Assessment Topics}
\begin{center}
\begin{tabular}{cl|c|c|c|} \\ \hline
\multicolumn{1}{|c|}{\textbf{S\slash n}} & \multicolumn{1}{c|}{\textbf{Topic(s)}} & \textbf{Form(s)} & \textbf{Pts} & \textbf{Total} \\ \hline \hline
\multicolumn{1}{|c|}{1} &  & I & 10 & \multirow{10}{*}{100} \\ \cline{1-4}
\multicolumn{1}{|c|}{2} &  & II & 10 & \\ \cline{1-4}
\multicolumn{1}{|c|}{3} &  & II & 10 & \\ \cline{1-4}
\multicolumn{1}{|c|}{4} &  & IV & 10 & \\ \cline{1-4}
\multicolumn{1}{|c|}{5} &  & I\slash II & 10 & \\ \cline{1-4}
\multicolumn{1}{|c|}{6} &  & I\slash III & 10 & \\ \cline{1-4}
\multicolumn{1}{|c|}{7} &  & I & 10 & \\ \cline{1-4}
\multicolumn{1}{|c|}{8} &  & III & 10 & \\ \cline{1-4}
\multicolumn{1}{|c|}{9} &  & II\slash IV & 10 & \\ \cline{1-4}
\multicolumn{1}{|c|}{10} &  & II & 10 & \\ \hline
\end{tabular}
\end{center}

%\subsubsection{Assessment Topics}
%\noindent The following table shows the topics assessed in the FTNA as well as the percentage they have appeared in past examinations. 
%\begin{center}
%	\begin{tabular}{|c|l|c|} \hline
%		Form & \multicolumn{1}{|c|}{Topic} & Percentage [\%] \\ \hline
%		\multirow{11}{*}{I} 	& Numbers				& \\ \cline{2-3}
%						& Fractions				& \\ \cline{2-3}
%						& Decimals and Percentages	& \\ \cline{2-3}
%						& Units					& \\ \cline{2-3}
%						& Approximations		 	& \\ \cline{2-3}
%						& Geometry 				& \\ \cline{2-3}
%						& Algebra					& \\ \cline{2-3}
%						& Ratio, Profit and Loss		& \\ \cline{2-3}
%						& Coordinate Geometry		& \\ \cline{2-3}
%						& Perimeters and Areas		& \\ \cline{2-3}
%						& Exponents and Radicals	& \\ \hline
%		\multirow{9}{*}{II} 	& Quadratic Equations		& \\ \cline{2-3}
%						& Logarithms				& \\ \cline{2-3}
%						& Congruence				& \\ \cline{2-3}
%						& Similarity				& \\ \cline{2-3}
%						& Geometrical Transformations	& \\ \cline{2-3}
%						& Pythagoras Theorem		& \\ \cline{2-3}
%						& Trigonometry				& \\ \cline{2-3}
%						& Sets					& \\ \cline{2-3}
%						& Statistics				& \\ \hline
%	\end{tabular}
%\end{center}

%%%%%%%%%%%%%%%%%%%%%%%%%%%%%
%-----------------------------Form IV------------------------------------
%%%%%%%%%%%%%%%%%%%%%%%%%%%%%
\subsection{Form IV}
\noindent The following format for the Form Four Certificate of Secondary Education (CSEE) is based on the revised version of the Basic Mathematics Syllabus for Ordinary Secondary Education of 2005. The exam is intended to assess the competences acquired by the students after four years of study. The following information was taken from The National Examinations Council of Tanzania - Certificate of Secondary Education Examination Formats %CITE!!!!!!!!!!!!!!!!!!

\subsubsection{General Objectives}
\noindent The general objectives of the Basic Mathematics assessment are to test students' ability to:
\begin{enumerate}[topsep=1ex,itemsep=0ex,partopsep=1ex,parsep=1ex]
	\item Mathematical competences among candidates in solving practical problems in daily life have been developed
	\item Mathematical concepts can be applied by candidates in interpreting situations at local and global levels
	\item Candidates are able to use mathematical knowledge, techniques and life skills for studying mathematics and related subjects
	\item Candidates have been adequately prepared for higher studies
\end{enumerate}

\subsubsection{General Competences}
\noindent The FTNA will specifically test the students' ability to:
\begin{enumerate}[topsep=1ex,itemsep=0ex,partopsep=1ex,parsep=1ex]
	\item Think critically and logically in interpreting and solving problems
	\item Use mathematical language in explaining and clarifying mathematical ideas
	\item Apply mathematical knowledge and techniques in other fields
\end{enumerate}

\subsubsection{Examination Rubric}
\noindent The examination consists of \textbf{one (1)} theory paper. The duration of the exam is \textbf{3 hours}. The paper consists of \textbf{sixteen (16)} short answer questions categorized into section A and B. Section A consists of \textbf{ten (10)}questions (6 marks each). Section B has \textbf{six (6)} questions (10 marks each). The students are required to attempt \textbf{all} questions in section A and \textbf{four (4)} in section B, clearly showing all workings. \\

\noindent Unlike other subjects present in this manual, the Basic Mathematics examination follows a special format. The format requires specific questions to cover specific topics (see table below). For example, Question 2 will \emph{always} be about exponents, radicals, and/or logarithms. Similarly, Question 11 will \emph{always} cover the topic of linear programming. Thus, it is possible for a student to effectively pass their exam, even by knowing just \emph{two} topics very well (e.g Statistics and Accounts). Of course, it is not necessarily recommended to have students only focus on learning a few select topics, as this does not encourage thorough learning of the material, but it can be used as a motivator to students who have already written themselves off for the Math NECTA.

\begin{center}
\begin{tabular}{cl|c|c|c|} \\ \hline
\multicolumn{1}{|c|}{\textbf{S\slash n}} & \multicolumn{1}{c|}{\textbf{Topic(s)}} & \textbf{Form(s)} & \textbf{Pts} & \textbf{Total} \\ \hline \hline
\multicolumn{5}{|c|}{\textbf{SECTION A (60 MARKS)}} \\ \hline
\multicolumn{1}{|c|}{1} & Numbers\slash Fractions\slash Decimals and Percentages\slash Approximations & I & 6 & \multirow{10}{*}{60} \\ \cline{1-4}
\multicolumn{1}{|c|}{2} & Exponents\slash Radicals\slash Logarithms & II & 6 & \\ \cline{1-4}
\multicolumn{1}{|c|}{3} & Algebra\slash Sets & II & 6 & \\ \cline{1-4}
\multicolumn{1}{|c|}{4} & Coordinate Geometry\slash Vectors & IV & 6 & \\ \cline{1-4}
\multicolumn{1}{|c|}{5} & Geometry\slash Perimeter and Area\slash Congruence\slash Similarity & I\slash II & 6 & \\ \cline{1-4}
\multicolumn{1}{|c|}{6} & Units\slash Rates and Variations & I\slash III & 6 & \\ \cline{1-4}
\multicolumn{1}{|c|}{7} & Ratio, Profit and Loss & I & 6 & \\ \cline{1-4}
\multicolumn{1}{|c|}{8} & Sequences and Series & III & 6 & \\ \cline{1-4}
\multicolumn{1}{|c|}{9} & Trigonometry\slash Pythagoras Theorem & II\slash IV & 6 & \\ \cline{1-4}
\multicolumn{1}{|c|}{10} & Quadratic Equations & II & 6 & \\ \hline
\multicolumn{5}{|c|}{\textbf{SECTION B (40 MARKS - CHOOSE 4)}} \\ \hline
\multicolumn{1}{|c|}{11} & Linear Programming & IV & 10 & \multirow{6}{*}{40} \\ \cline{1-4}
\multicolumn{1}{|c|}{12} & Statistics & III & 10 & \\ \cline{1-4}
\multicolumn{1}{|c|}{13} & Three Dimensional Figures\slash Circles\slash Earth as a Sphere & III\slash IV & 10 & \\ \cline{1-4}
\multicolumn{1}{|c|}{14} & Accounts & III & 10 & \\ \cline{1-4}
\multicolumn{1}{|c|}{15} & Matrices and Transformations & IV & 10 & \\ \cline{1-4}
\multicolumn{1}{|c|}{16} & Probability\slash Functions\slash Relations & III\slash IV & 10 & \\ \hline

\end{tabular}
\end{center}

%Letter grades are assigned based on the following scale for Form IV NECTA exams:
%\begin{center}
%\begin{tabular}{c|c}
%Score & Grade \\ \hline
%81 - 100 & A \\
%61 - 80 & B \\
%41 - 60 & C \\
%21 - 40 & D \\
%0 - 20 & F \\
%\end{tabular}
%\end{center}
%
%Have students plan ahead of time to formulate a strategy for the exam. Which 4 topics does each student feel most comfortable with from Section B? Students should have 1 fall-back topic in case one of their preferred topics is particularly challenging on an exam. Advise students to begin with problems from Section B, as these carry more marks and therefore should have more time dedicated to them. Which topics from Section A are strengths for a student, and which ones are weaknesses? Students should answer their strong topics first, and use whatever time they have remaining for weaker topics.
%
%Planning ahead and utilizing a strategy for taking the NECTA exam will give students added confidence in their test-taking abilities and will also give them a strategy for what topics to study in most detail leading up to the exam. Make sure to give students plenty of practice exams before the actual NECTA! Many students only have one opportunity to familiarize themselves with the layout of the test during Mock Examinations a few months ahead of time, but this is not enough. Comfort and confidence in one's math abilities come from practice, practice and more practice!

