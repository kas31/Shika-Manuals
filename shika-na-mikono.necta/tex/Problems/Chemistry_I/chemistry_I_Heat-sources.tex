\subsection{Heat Sources}

\begin{enumerate}
	\item List four properties of each of the following: A luminous flame, A non-luminous flame
	
	\item A non-luminous flame is obtained if the air hole is:
	\begin{enumerate}[topsep=0ex,itemsep=0ex,partopsep=1ex,parsep=1ex]
		\item[(A)] Fully opened
		\item[(B)] Partially open
		\item[(C)] Closed
		\item[(D)] Half opened
	\end{enumerate}
	
	\item Fill in the blanks: A flame is a zone burning gas that produces \rule{1.5cm}{0.15mm} and \rule{1.5cm}{0.15mm}.

	\item Why a flame produced by a "spirit lamp`` may not be good for heating in the laboratory? Give two reasons.
	
	\item Name the type of flame produced by a spirit lamp.
%%%%%%%%%%%%%%%%%%%%%%%%%%%%%%%%%%%%%%%%%%%%%%%%%%%%%%%%%%%	
	\item Technicians prefer to use blue flame in welding because:
	\begin{enumerate}[topsep=0ex,itemsep=0ex,partopsep=1ex,parsep=1ex]
		\item[(A)] It is bright and non-sooty
		\item[(B)] It is light and non-sooty
		\item[(C)] It is very hot and large
		\item[(D)] It is very hot and non-sooty
		\item[(E)] It is not expensive
	\end{enumerate}
\end{enumerate}