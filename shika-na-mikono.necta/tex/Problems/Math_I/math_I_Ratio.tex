\subsection{Ratio, Profit and Loss}
\begin{enumerate}

		\subsubsection{Ratio}
	\item An alloy consists of three metals A, B and C in the proportions $A:B = 3:5$ and $B:C = 7:6$. Calculate the proportion $A:C$.		
		
	\item Given the ratios:\\
	$A : B = 2 : 3$\\
	$B : C = 6 : 7$\\
	Calculate the ratio of $A : C$.
	
	\item Express $2\frac{1}{2} : 3$ as integers in a simplified form.
	
	\item If it is known that $x:y = 5:1$ find the value of $\cfrac{x + y}{3x - 4y}$.
	
	\item Three numbers $d$, $m$ and $n$ are in the ratio of $3:6:4$ respectively. Find the value of $\cfrac{4d - m}{m + 2n}$.
	
	\item A, B and C are to share Tshs. 120,000/= in the ratio $2:3:5$ respectively. How much will each get?
	
	\item Three people share a property in the ratio $2:x:y$. It is known that $y = x + 2$. If the largest shareholder had Tsh. 39,100/= in monetary terms, find the value of this property.
	
	\item The ratio of men : women : children living in Mkuza village is 6 : 7 : 3. If there are 42,000 women, find how many:
		\begin{itemize}
		\item[(a)]
			\begin{itemize}
			\item[(i)] children live in Mkuza village.
			\item[(ii)] people altogether live in Mkuza village.
			\end{itemize}
		\item[(b)] The 42,000 women is an increase of 20\% on the number of women ten years ago. How many women lived in the village?
		\end{itemize}
		
	\item An amount of Tshs. 12,000 is to be shared among Ali, Anna and Juma in the ratio 2 : 3 : 5 respectively. How much will each get?

	\item The are of two circles are in the ratio of 16 : 9. Calculate the radius of the smaller circle when the radius of the larger one is 24 cm.
	
	\item The ratio of the areas of two circles is 50 : 72. If the radius of the smaller circle is 15 cm, find the radius of the larger circle.
	
	\item The sides of a rectangle are in the ratio 3 : 5. If the perimeter of this rectangle is 800 cm, find the dimensions of the rectangle.
	
	\item The distance between two towns on a map of scale 1 : 5,000,000 is 9 cm. Find the actual distance between the towns in kilometres.
	
	\item A building 250 metres high is represented by a line segment of length 5 cm. Find the scale of drawing.
	
	\item An area of 24.7 cm$^2$ was plotted on a map of a scale 1 : 50,000. What was this area, in square kilometres, on the Earth's surface?
	
	
		\subsubsection{Profit and Loss}
%profit and loss
	\item 
		\begin{itemize}
		\item[(a)] John and Paul started a tailoring business and invested shs 110,250/= and 220,500/= respectively. If the profit after the first six months was shs 50,970/=, how much did Paul get if they agreed to share it according to the amount they invested.
		\item[(b)] Mary paid shs 800,000/= for a computer and sold it the following year for shs 600,000/=. Find the percentage loss she got.
		\end{itemize}
		
	\item By selling an article at shs. 22,500/= a shopkeeper makes a loss of 10\%. At what price must the shopkeeper sell the article in order to get a profit of 10\%?
	
	\item Neema bought a tray of eggs (containing 30 eggs) for shs. 2,000/=. She boiled the eggs using a litre of kerosene costing shs. 400/= and sold each egg at a price of 100/= each. Find her percentage profit.
	
	\item A shopkeeper sells sugar at sh. 105.00 per kilogram. If he realizes a profit of 5\% over the buying price, find the buying price per kilogram.
	
	
		\subsubsection{Simple Interest}
%simple interest	
	\item Find the amount of money obtained after depositing 900/= for 2 years and 9 months at an annual rate of 6\% simple interest.
	
	\item How long will it take a sum of money to double itself at 5\% per months, simple interest?
	
	\item A certain amount of money was deposited for three years at the annual rate of 5\% simple interest. The interest at the end of the three years was shs. 112.50. Find the principal.
	
	\item For how long should Tshs. 1,200,000/= be invested at simple interest rate of 5\% to get an interest of Tshs. 180,000/=?
	
	\item Mavuno wants to invest lump sum money so that its value after 4 years will be 812,000/=. How much should the investor invest at 4\% per annum single interest?

\end{enumerate}
