\section{Introduction to Chemistry Practicals} \label{chem-practical-format}

%----------------------------------------------------------------------------------------------------------------------------
\subsection{Format}
The theory portion of the Chemistry exam comprises 100 marks, while the practical carries 50 marks. A student's final grade for Chemistry is thus found by taking her total marks from both exams out of 150.

As of now, the Chemistry practical has 3 questions and students must answer \textbf{all} of them. Question 1 is on Volumetric Analysis and Laboratory Techniques and Safety. Question 2 is taken from Ionic Theory and Electrolysis\slash Chemical Kinetics, Equilibrium and Energy. Question 3 is on Qualitative Analysis. Question 1 is worth 20 marks, while Questions 2 and 3 carry 15 marks each. Students have 2$\frac{1}{2}$ hours to complete the exam.

Students are allowed to use Qualitative Analysis guidesheet pamphlets in the examination room.

%----------------------------------------------------------------------------------------------------------------------------
\subsection{NECTA Advance Instructions for Teachers}
There are two sets of advance instructions. One set of advance instructions are given to teachers at least one month before the date of the exam. These instructions contain the list of apparatus, chemicals, and other materials required for preparing the Chemistry practical questions. The instructions also give suggestions on the amount of chemicals that should be available for each candidate to use.

The second set of instructions should be given 24 hours before the time of the practical. It includes which chemicals and apparatus should be given to each candidate (or shared among candidates) for each of the three practical questions. These instructions also state how to label each solution and/or compound.

The bottom of the 24 Hours Advance Instructions also states that the Laboratory Technician or Head of Chemistry Department should perform some of the experiments immediately after the last session of the examination. It is only required to perform the titration and chemical kinetics experiments. This is required to be done for every school and is used as a reference for the markers in case the water, chemicals, and apparatus are not the same at every school. This is enclosed and submitted together with the students' test papers and may be used as a marking scheme. It is also advised that any notes, comments or concerns for the markers be included at this time.

%----------------------------------------------------------------------------------------------------------------------------
\subsection{Common Practicals}
\begin{description}
\item[Volumetric Analysis]{determine the concentration of a solution of a known chemical by reacting it with a known concentration of another solution}
\item[Qualitative Analysis]{systematically identify an unknown salt through a series of chemical tests}
\item[Chemical Kinetics and Equilibrium]{observe changes in chemical reaction rates by varying conditions such as temperature and concentration}
\end{description}

\paragraph{Note} These are the most common practicals, but they are not necessarily the only practicals that can occur on a NECTA exam. Although the updated exam format lists Questions 1 and 3 as Volumetric Analysis and Qualitative Analysis respectively, Question 2 can come from a variety of topics which may not yet have been used in older past papers. Be sure to regularly check the most recent past NECTA papers to get a good idea of the types of questions to expect. 

%==============================================================================

\section{Volumetric Analysis}

This section contains the following: 

\begin{itemize}[topsep=0ex,itemsep=0ex,partopsep=1ex,parsep=1ex]
	\item Volumetric Analysis Theory
	\item Traditional Volumetric Analysis Technique
	\item Common Calculations in Titration Experiments
	\item Tips and Tricks
	\item Sample Practical Questions
\end{itemize}

%----------------------------------------------------------------------------------------------------------------------------
\subsection{Volumetric Analysis Theory}

\noindent Volumetric Analysis is a method to find the concentration (molarity) of a solution of a known chemical by comparing it with the known concentration of a solution of another chemical known to react with the first. For example, to find the concentration of a solution of citric acid, one might use a 0.1~M solution of sodium hydroxide because sodium hydroxide is known to react with citric acid. \\

\noindent The most common kinds of volumetric analysis are for acid-base reactions and oxidation-reduction reactions. Acid-base reactions require use of an indicator, a chemical that changes color at a known pH. Some oxidation-reduction reactions require an indicator, often starch solution, although many are self-indicating, (one of the chemicals itself has a color). \\

\noindent The process of volumetric analysis is often called \textit{titration.}

%----------------------------------------------------------------------------------------------------------------------------
\subsection{Traditional Volumetric Analysis Technique}

\noindent The Volumetric Analysis practical consists of an acid that is being titrated acid against a base until neutralization, in order to determine the concentration of the base. On NECTA practical exams, titrations are done four times: a pilot followed by three trials. The pilot is done quickly and is used to determine the approximate volume needed for neutralization to speed up the following trials.\\

\noindent Ex: If the pilot gives an end point of 25.00 mL, then for the three subsequent trials, 20.00 mL can quickly be added from the burette. Then begin to add solution slowly until the endpoint is reached.

\paragraph{Note} Results from the pilot are not accurate and are not included when doing calculations. Students should also know that not all three trials are always used in calculating the average volume used. Values of trials must be consistent and within $\pm$ 02 cm$^3$ of each other to be valid for average volume determination.

%----------------------------------------------------------------------------------------------------------------------------
\subsubsection{Volumetric Analysis Using Burettes}

\paragraph{Preparation}

\begin{enumerate}
\item After washing Burettes thoroughly, rinse the Burette with 3 mL of the acidic solution that will
be used during the titration (Acid usually goes in the Burette).
	\begin{itemize}[topsep=0ex,itemsep=0ex,partopsep=1ex,parsep=1ex]
	\item Cover the entire inside surface of the Burette.
	\item Discard 3 mL of solution properly when finished  .
	\item Why? This prevents dilution of acid by water.
	\end{itemize}
\item After washing the flask thoroughly, rinse the flask with 3 mL of solution that will
be used during the titration (Base usually goes in the flask).
	\begin{itemize}[topsep=0ex,itemsep=0ex,partopsep=1ex,parsep=1ex]
	\item Cover the entire inside surface of the flask.
	\item Discard 3 mL of solution properly when finished.
	\end{itemize}	
\end{enumerate}

\paragraph{Procedure}
\begin{enumerate}[topsep=0ex,itemsep=0ex,partopsep=1ex,parsep=1ex]
\item Clean the burette with water.
\item Rise the burette with the acid that will be used for the titration.
\item Fill the burette with the acid. Let a little run out through the stopcock.
\item Record the initial burette reading.
\item Use a syringe to transfer the base solution into a conical flask.
\item Record the volume moved by the syringe.
\item If you are using an indicator, add a few drops to the flask.
\item Slowly add the acid from the burette to the flask. Swirl the flask as you titrate. Be careful. Avoid acid drops landing on the sides of the flask.
\item Stop titration when the slight color change become permanent. This is the end point.
\item Record final reading of the burette.
\item Repeat for remaining titrations.
\end{enumerate}

\paragraph{Notes}
\begin{itemize}
\item Burettes tell you the volume of solution used, not the volume present.\\
\textbf{Ex:} Initial Reading - 4.23 mL\\
Final Reading - 20.57 mL\\
You used 16.34 mL of acid during the titration.

\item Reading Measurements
	\begin{itemize}[topsep=0ex,itemsep=0ex,partopsep=1ex,parsep=1ex]
	\item Always read burettes at eye level.
	\item Always read from the bottom of the meniscus. In a plastic apparatus, there is often no meniscus.
	\item Burettes are accurate to 2 decimal places. Students should estimate to the nearest 0.01 mL
	\end{itemize}

\item For Acid-Base indicators: The less indicator used, the better. To change color, the indicator must react with fluid in the burette. If you add too much, it uses more chemical than necessary for neutralization, creating an indicator error.
		
\item For starch indicators: use 1 mL. Starch is not titrated; indicators are, and you must use more to get a good color change.

\end{itemize}

%----------------------------------------------------------------------------------------------------------------------------
\subsection{Common Calculations in Titration Experiments}
\label{sub:titcalc}

All NECTA practical experiments require students to determine some unknown in the titration procedure. Common calculations that the problem statement will ask for include:

\begin{itemize}[topsep=0ex,itemsep=0ex,partopsep=1ex,parsep=1ex]

\item{Concentration (molarity) of an acid or base}
\item{Relative atomic mass of unknown elements in an acid or base}
\item{Percentage purity of a substance}
\item{Amount of water of crystallization in a substance}

\end{itemize}

%----------------------------------------------------------------------------------------------------------------------------
\subsubsection{Concentration of an Acid or Base}

The problem statement may have the student find either the unknown molarity (moles per litre) or concentration (grams per litre) of the acid or the base. As an example, the following steps are used to calculate the unknown concentration of an acid:

\begin{enumerate}
\item[1.] \textit{Calculate the average volume of acid used.}\\
Remember to not use the pilot trial or any trials that are not within $\pm$ 0.2 cm$^3$ of each other.
\item[2.] \textit{Calculate the number of moles of the base used.}\\
$$\text{Molarity} = \frac{\text{number of moles}}{\text{volume of solution}}$$\\
These values can usually be taken from the solutions listed on the test paper. Also be sure that the units of volume of solution are in litres or dm$^3$.
\item[3.] \textit{Write a balanced chemical equation for the reaction.}\\
The chemical equation can also be written as an ionic equation.
\item[4.] \textit{Calculate the number of moles of acid used from the mole ratio taken from the balanced chemical equation.}\\
Both ionic and full formulae equations give the same mole ratio.
\item[5.] \textit{Work out the molar concentration of the acid.}\\
The molar concentration can be determined using the calculated number of moles of acid (found in the previous step) and the average volume of acid used (found in step 1), using the equation in step 2.\\
Alternatively, the following equation can be used:\\
$$\cfrac{C_AV_A}{C_BV_B} = \cfrac{n_A}{n_B}$$\\
where:\\
C$_A$ is the molar concentration of the acid,\\
V$_A$ is the volume of the acid used,\\
\textit{n}$_A$ is the number of moles of the acid used,\\
C$_B$ is the molar concentration of the base,\\
V$_B$ is the volume of the base used, and\\
\textit{n}$_B$ is the number of moles of the base used.\\
\end{enumerate}
Similar steps are used to calculate the unknown concentration of a base.\\
Repeat steps 1 through 5, but with the following changes:\\
\begin{itemize}[topsep=0ex,itemsep=0ex,partopsep=1ex,parsep=1ex]
\item{Step 2: Calculate the moles of the acid used.}
\item{Step 4: Calculate the moles of the base from the mole ratio.}
\item{Step 5: Find the molar concentration of the base, either using the molarity calculation or the equation above.}
\end{itemize}

%----------------------------------------------------------------------------------------------------------------------------
\subsubsection{Relative Atomic Mass of Unknown Elements}

Atomic mass of unknown elements, as well as molecular mass of compounds with unknown elements may need to be calculated in the problem statement. Most unknown elements will be a metal of a basic compound. As an example, the following steps are used to calculate the relative atomic mass of an unknown metal element of a metal carbonate:

\begin{enumerate}
\item[1.] \textit{Calculate the average volume of acid used.}\\
Remember to not use the pilot trial or any trials that are not within $\pm$ 0.2 cm$^3$ of each other.
\item[2.] \textit{Calculate the number of moles of the acid used.}\\
$$\text{Molarity} = \frac{\text{number of moles}}{\text{volume of solution}}$$\\
These values can usually be taken from the solutions listed on the test paper. Also be sure that the units of volume of solution are in litres or dm$^3$.
\item[3.] \textit{Write a balanced chemical equation for the reaction to get the mole ratio.}
\item[4.] \textit{Determine the number of moles of the metal carbonate used.}\\
This can be taken from the balanced chemical equation.
\item[5.] \textit{Work out the molecular concentration of the metal carbonate solution.}\\
Use the formula as shown in step 2.
\item[6.] \textit{Calculate the mass of the metal carbonate in one litre of solution.}\\
This can be done using the following ratio:\\
$$\frac{\text{mass given in problem statement}}{\text{volume given in problem statement}} = \frac{\text{mass of unknown metal}}{\text{one litre}}$$\\
Make sure the units correspond because sometimes the problem statement will be expressed in dm$^3$ or cm$^3$.
\item[7.] \textit{Using the molarity of the solution and the mass of the metal carbonate per litre of solution, work out the relative molecular mass of the metal carbonate.}\\
The following equation can be used to calculate molar mass:\\
$$\text{molar mass} = \frac{\text{mass per litre}}{\text{molarity}}$$
\item[8.] \textit{Calculate the relative atomic mass of the metal based on the formula of the carbonate.}\\
Use the total molar mass of the compound found in step 7 and the molar mass of each element in the compound to find the molar mass of the unknown element.
\end{enumerate}
Some problem statements may require the student to identify the unknown element from its molecular mass.

\paragraph{Note} Similar steps should be followed if the unknown element is of an acidic compound. Just replace the steps that include the metal carbonate solution with the acid solution.

%----------------------------------------------------------------------------------------------------------------------------
\subsubsection{Percentage Purity of a Substance}

Problem statements that require the student to find percentage purity will usually contain one solution in the list provided that specifically states it is impure or that it is a hydrated compound (seems very low in concentration). Again, it is possible to determine percentage purity of an acid or a base. As an example, the following steps are used to calculate the percentage purity of a base:

\begin{enumerate}
\item[1.] \textit{Determine the average volume of the acid used.}\\
Remember to not use the pilot trial or any trials that are not within $\pm$ 0.2 cm$^3$ of each other.
\item[2.] \textit{Calculate the number of moles of the acid used.}\\
$$\text{Molarity} = \frac{\text{number of moles}}{\text{volume of solution}}$$\\
These values can usually be taken from the solutions listed on the test paper. Also be sure that the units of volume of solution are in litres or dm$^3$.
\item[3.] \textit{Write a balanced chemical equation for the reaction to get the mole ratio.}
\item[4.] \textit{Determine the number of moles of base used in the reaction.}\\
This can be taken from the mole ratio from the previous step.
\item[5.] \textit{Calculate the mass of the base used in the reaction.}\\
The mass can be determined by the number of moles calculated and the following relationship:\\
$$\text{mass} = \text{number of moles} \times \text{molar mass}$$
\item[6.] \textit{Work out the percentage purity of the base solution sample.}\\
The following equation for percentage purity should be used:\\
$$\text{percentage purity} = \frac{\text{mass of pure substance in sample}}{\text{mass of the impure sample}} \times 100\%$$\\
It is very important to note that when calculating percentage purity, the amount of volume in the concentration of base must be equal to the volume of concentration of acid used. For example, if there was 0.424 g of sodium carbonate in 25 cm$^3$ of solution reacting with a 250 cm$^3$ solution of acid, the mass of sodium carbonate must be converted to know the mass in 250 cm$^3$.  Therefore, 250 cm$^3$ of base solution will contain 4.24 g, not 0.424 g.\\
The value for the mass of the impure sample comes from the list of provided solutions and the mass of the pure sample will come from the calculations.
\end{enumerate}

\paragraph{Note} Similar steps can be followed to find the percentage purity of a the acid solution sample. Instead of finding the mass of the base, use the calculated moles of acid used to find the mass of acid in the actual reaction.

%----------------------------------------------------------------------------------------------------------------------------
\subsubsection{Amount of Water of Crystallization}

Water of crystallization is the water that is bound within crystals of substances. Most hydrated substances and solutions contain water of crystallization. Problem statements that ask students to determine the amount of water of crystallization will have a solution with a formula similar to [base]$\cdot$xH$_2$O, and they have to solve for x. As an example, the following steps are used to determine the number of molecules of water of crystallization in a hydrated base compound sample:

\begin{enumerate}
\item[1.] \textit{Calculate the average volume of the acid used.}\\
Remember to not use the pilot trial or any trials that are not within $\pm$ 0.2 cm$^3$ of each other.
\item[2.] \textit{Calculate the number of moles of the acid used.}\\
$$\text{Molarity} = \frac{\text{number of moles}}{\text{volume of solution}}$$\\
These values can usually be taken from the solutions listed on the test paper. Also be sure that the units of volume of solution are in litres or dm$^3$.
\item[3.] \textit{Write a balanced chemical equation for the reaction to get the mole ratio.}
\item[4.] \textit{Calculate the number of moles of the base used.}\\
This can be determined from the mole ratio in the previous step.
\item[5.] \textit{Determine the molar concentration of the base.}\\
The molarity can be calculated using the volume of base used in the experiment and the equation from step 2.
\item[6.] \textit{Calculate the relative molecular mass (R.M.M.) of the base compound.}\\
The following equation can be used to calculate molar mass:\\
$$\text{molar mass} = \frac{\text{mass per litre}}{\text{molarity}}$$
\item[7.] \textit{Determine the number of molecules of water of crystallization in the sample.}\\
Using the relative atomic masses of the various atoms in the base compound, subtract the mass of the compound from the total mass of the hydrated compound. Water molecules always have a total molecular mass of 18 g/mol, so the remaining mass will be composed of multiples of 18. For example, if a hydrated carbonate (Na$_2$CO$_3$ $\cdot$xH$_2$O) has a total mass of 286 g, the molecules of water can be determined as follows:\\
$2\text{Na} + \text{C} + 3\text{O} + x(2\text{H} + \text{O}) = 286$\\
$(2 \times 23) + 12 + (3 \times 16) + x[(2 \times 1) + 16] = 286$\\
$106 + 18x = 286$\\
$18x = 180$\\
$x = 10$\\

Therefore, in this example, there are 10 molecules of water of crystallization in the hydrated sodium carbonate (Na$_2$CO$_3$ $\cdot10$H$_2$O) sample.
\end{enumerate}

%----------------------------------------------------------------------------------------------------------------------------
\subsection{Tips and Tricks}
\noindent The following are some tips and tricks for successfully performing the Volumetric Analysis practical:
\begin{itemize}[topsep=0ex,itemsep=0ex,partopsep=1ex,parsep=1ex]
	\item Create a table for the titration values (should contain the pilot and the three trials)
	\item Put 25 mL of the acid or base that will be in the flask
	\item Titrate until you see a PERMANENT color change
	\item Do a pilot titration to know about how many mL you will need
	\item For titrations after the pilot, when you get close to the pilot value add the acid or base SLOWLY and mix between drops
	\item Swirl flask to mix them thoroughly
	\item Only use values between $\pm$0.02, and do not use the pilot values in calculations
\end{itemize}
%----------------------------------------------------------------------------------------------------------------------------
\subsection{Sample Practical Questions}
%The following is a sample practical question from 2012.\\
%
%
%\noindent You are provided with the following solution:\\
%
%\noindent \textbf{TZ}: Containing 3.5 g of impure sulphuric acid in 500 cm$^3$ of solution;\\
%\textbf{LO}: Containing 4 g of sodium hydroxide in 1000 cm$^3$ of solution;\\
%Phenolphthalein and Methyl indicators.\\
%
%\textbf{Questions}:\\
%\begin{enumerate}
%\item[(a)]
%\begin{enumerate}
%\item[(i)] What is the suitable indicator for the titration of the given solutions?\\
%Give a reason for your answer.
%\item[(ii)] Write a balanced chemical equation for the reaction between \textbf{TZ} and \textbf{LO}.
%\item[(iii)] Why is it important to swirl or shake the contents of the flask during the addition of the acid?\\
%\end{enumerate}
%
%\item[(b)] Titrate the acid (in a burette) against the base (in a conical flask) using two drops of your indicator and obtain three titre values.\\
%
%\item[(c)] 
%\begin{enumerate}
%\item[(i)] \_\_\_\_ cm$^3$ of acid required \_\_\_\_ cm$^3$ of base for complete reaction.
%\item[(ii)] Showing your procedures clearly, calculate the percentage purity of \textbf{TZ}. \hfill \textbf{(20 marks)}
%\end{enumerate}
%\end{enumerate}

The following are two sample practical questions:
\subsubsection{Sample Practical \#1}

\noindent You are provided with the following:\\
\noindent \textbf{PP:} A solution of 0.1 M hydrchloric acid; \\
\noindent \textbf{RR:} A solution of 1.39 g of impure sodium carbonate anhydrous dissolved in 250 cm$^3$ of solution; \\
\noindent \textbf{MO:} Methyl orange indicator \\

\noindent \textbf{Questions:} 
\begin{enumerate}[topsep=0ex,itemsep=0ex,partopsep=1ex,parsep=1ex]
	\item[(a)] 
	\begin{enumerate}[topsep=0ex,itemsep=0ex,partopsep=1ex,parsep=1ex]
		\item[i)] Titrate solution PP against 20 cm$^3$ or 25 cm$^3$ of RR until the colour change. Record the burette ratings. Repeat the procedure to obtain three accurate readings and record your results in a tabular form.
		\item[ii)] Why did the colour of the solution change?
		\item[iii)] Determine the average title volume
		\item[iv)] \rule{1.5cm}{0.15mm} cm$^3$ of solution RR required \rule{1.5cm}{0.15mm} cm$^3$ of solution PP for a complete reaction
		\item[v)] Assume that sulphuric acid of the same molarity was used in the place of hydrochloric acid, would it be a difference in the title volume used? Give reason.
	\end{enumerate}
	\item[(b)]
	\begin{enumerate}[topsep=0ex,itemsep=0ex,partopsep=1ex,parsep=1ex]
		\item[i)] Name the apparatus you used for measuring volume of PP
		\item[ii)] Why the apparatus in (b) i) is the best recommended for its function?
	\end{enumerate}
	\item[(c)] Write a balanced chemical equation of the reaction between the solutions PP and RR
	\item[(d)] Calculate the following and write your answer in two decimal places
	\begin{enumerate}[topsep=0ex,itemsep=0ex,partopsep=1ex,parsep=1ex]
		\item[i)] molarity of RR
		\item[ii)] percentage purity of RR
	\end{enumerate}
	\item[(e)] State two applications of volumetric analysis
\end{enumerate}

\subsubsection{Sample Practical \#2}

\noindent You are provided with the following:\\
\noindent \textbf{S:} A solution containing 0.125 M sulphuric acid (H$_2$SO$_4$); \\
\noindent \textbf{T:} A solution made by dissolving 15 g of impure sodium hydroxide (NaOH) in distilled water making up to 1000 cm$^3$ of the solution; \\
\noindent \textbf{MO:} Methyl orange indicator \\

\noindent \textbf{Questions:} 
\begin{enumerate}[topsep=0ex,itemsep=0ex,partopsep=1ex,parsep=1ex]
	\item[(a)] 
	\begin{enumerate}[topsep=0ex,itemsep=0ex,partopsep=1ex,parsep=1ex]
		\item[i)] Titrate S (from the burette) against T (in the titration flask) using MO up to the end point. Repeat the procedure to obtain three accurate readings and record your results in a tabular form. 
		\item[ii)] Calculate average volume of S used
		\item[iii)] \rule{1.5cm}{0.15mm} cm$^3$ of T required \rule{1.5cm}{0.15mm} cm$^3$ of S for a complete reaction
		\item[iv)] The color changed at the neutralization point was from \rule{1.5cm}{0.15mm} to \rule{1.5cm}{0.15mm}.
	\end{enumerate}
	\item[(b)] Write a balanced chemical equation for the reaction taking place between S and T.
	\item[(c)] Calculate the percentage purity and percentage impurity of sodium hydroxide
\end{enumerate}

%==============================================================================
\clearpage
%==============================================================================

\section{Qualitative Analysis} \index{Practicals! Chemistry! qualitative analysis} \index{Qualitative analysis}
\label{cha:qualana}
%[brief 1 paragraph explanation of practical]\\

This section contains the following:
\begin{itemize}[topsep=0ex,itemsep=0ex,partopsep=1ex,parsep=1ex]
	\item Overview of Qualitative Analysis
	\item The Steps of Qualitative Analysis
	\item Tips and Tricks
	\item Sample Practical Questions
\end{itemize}

%----------------------------------------------------------------------------------------------------------------------------
\subsection{Overview of Qualitative Analysis}

The salts requiring identification have one cation and one anion. Generally, these are identified separately although often knowing one helps interpret the results of tests for the other. For ordinary level in Tanzania, students are confronted with binary salts made from the following ions:

\begin{itemize}
\item{Cations: NH$_{4}^{+}$, 
Ca$^{2+}$, 
Fe$^{2+}$, 
Fe$^{3+}$, 
Cu$^{2+}$, 
Zn$^{2+}$, 
Pb$^{2+}$, 
Na$^{+}$}
\item{Anions: CO$_{3}^{2-}$, 
HCO$_{3}^{-}$, 
NO$_{3}^{-}$, 
SO$_{4}^{2-}$, 
Cl$^{-}$}
\end{itemize}
At present, ordinary level students receive only one salt at a time. The teacher may also make use of qualitative analysis to identify unlabeled salts. 

%----------------------------------------------------------------------------------------------------------------------------
\subsection{The Steps of Qualitative Analysis}

\noindent The ions are identified by following a series of ten steps, divided into three stages. These are:
\begin{description}
	\item[Preliminary tests:] These tests use the solid salt. They are: appearance, action of heat, action of dilute H$_{2}$SO$_{4}$, action of concentrated H$_{2}$SO$_{4}$, flame test, and solubility.
	\item[Tests in solution:] The compound should be dissolved in water before carrying out these tests. If it is not soluble in water, use dilute acid (ideally \ce{HNO3}) to dissolve the compound. The tests in solution involve addition of NaOH and NH$_{3}$.
	\item[Confirmatory tests:] These tests confirm the conclusions students draw from the previous steps. By the time your students start the confirmatory tests, they should have a good idea which cation and which anion are present. Have students do one confirmatory test for the cation they believe is present, and one for the anion you believe is present. Even if several confirmatory tests are listed, students only need to do one. When identifying an unlabelled container, however, you might be moved to try several, especially if you are new to this process.
\end{description}

%%%%%%%%%%%%%%%%%%%%%%%%%%%%%%%%%%%%%%%%%%%%%%%%%%
%--------------------------------------------------PRELIMINARY TESTS--------------------------------------------------
%%%%%%%%%%%%%%%%%%%%%%%%%%%%%%%%%%%%%%%%%%%%%%%%%%
\subsubsection{PRELIMINARY TESTS}
\noindent The preliminary tests are generally for solid samples. As shown in Table \ref{chem_prelim}, the tests include appearance (colour, texture, and deliquescence), flame test, action of heat, solubility in water and action of dilute and concentrated acids. \\

\noindent [\textbf{Safety Precautions}: \textit{Avoid direct smelling of any chemical in the laboratory}] 

\begin{center}
	\begin{longtable}{|p{0.0125\textwidth}p{0.3375\textwidth}|p{0.31\textwidth}|p{0.32\textwidth}|} 		
	
	\multicolumn{4}{c}
	{{\bfseries \tablename\ \thetable{} -- Preliminary Tests}} \label{chem_prelim} \\
	\hline \multicolumn{2}{|p{0.35\textwidth}|}{\textbf{Experiment to be Performed}} & \multicolumn{1}{|p{0.31\textwidth}|}{\textbf{Expected Observations}} & \multicolumn{1}{|p{0.32\textwidth}|}{\textbf{Inference}} \\ \hline
	\endfirsthead
		
	\multicolumn{4}{c}
	{{\bfseries \tablename\ \thetable{} -- continued from previous page}} \\
	\hline \multicolumn{2}{|c|}{\textbf{Experiment to be Performed}} & \multicolumn{1}{|c|}{\textbf{Expected Observations}} & \multicolumn{1}{|c|}{\textbf{Inference}} \\ \hline
	\endhead
	
	\hline \multicolumn{4}{r}{{Continued on next page}} \\
	\endfoot
	
	\hline
	\endlastfoot
	
	%APPEARANCE OF SOLID SAMPLE
	1. & \textbf{Appearance of Solid Sample} (Texture and colour) & \multirow{5}{*}{White; crystalline or powder} & NH$_4^+$, Ca$^{2+}$, Zn$^{2+}$, Pb$^{2+}$ may be present \\ 
	& & & or \\
	& & & Transition metals Fe$^{2+}$, Fe$^{3+}$, Cu$^{2+}$ may be absent \\ \cline{3-4}
	
	& & Blue or green & Cu$^{2+}$ may be present \\ \cline{3-4}
	
	& & Pale or light green & Fe$^{2+}$ may be present \\ \cline{3-4}
	
	& & Yellowish or brownish & Fe$^{3+}$ may be present \\ \cline{3-4}
	
	& & White or coloured deliquescent crystals & \multirow{2}{*}{NO$_3^-$, Cl$^-$ may be present} \\ \hline
	
	%ACTION OF HEAT ON A SOLID SAMPLE
	2. & \textbf{Action of Heat on a Solid Sample} & White sublimate and a colorless gas evolve, which turn wet litmus paper from red to blue & \multirow{3}{*}{NH$_4^+$ may be present} \\ \cline{3-4}
	
	& [\textit{\textbf{Safety Precautions:} Hold the test-tube in a slanting position and away from observers and neighbours}] & Reddish brown fumes evolve and a gas which rekindles a glowing wooden splint & \multirow{3}{0.32\textwidth}{NO$_3^-$ may be present except those of Na$^+$ and of NH$_4^+$} \\ \cline{3-4}
	
	& & Colourless gas evolves, which relights a glowing splint & \multirow{2}{*}{NO$_3^-$ of Na$^+$ may be present} \\ \cline{3-4}
	
	& Transfer about 0.5 g of the solid sample in a dry test-tube. Heat strongly until no further change. Test for gas & Colourless gas evolves, which turns lime water milky and wet litmus paper from blue to red & \multirow{3}{*}{CO$_3^{2-}$, HCO$_3^-$ may be present} \\ \cline{3-4}
	
	& evolved & Colourless gas evolves, which turns filter paper dipped in acidified potassium dichromate solution from yellow to green & \multirow{4}{0.32\textwidth}{SO$_4^{2-}$ of Zn$^{2+}$, Fe$^{2+}$, Cu$^{2+}$ may be present} \\ \cline{3-4}
	
	& & Colourless gas evolves, which gives dense white fumes with ammonia solution & \multirow{3}{0.32\textwidth}{Cl$^-$ of hydrated Zn$^{2+}$, Cu$^{2+}$, Fe$^{2+}$, Fe$^{3+}$ salts may be present} \\ \cline{3-4}
	
	& & \multirow{4}{*}{No gas evolves} & SO$_4^{2-}$ of Na$^+$, Ca$^{2+}$, Pb$^{2+}$ may be present \\ \cline{4-4}
	
	& & & Cl$^-$ of Na$^+$, Pb$^{2+}$ may be present \\ \cline{4-4}
	
	& & & CO$_3^{2-}$ of Na$^+$ may be present \\ \cline{3-4} 
	
	& & Colourless droplets forming on the cooler parts of the test-tube, which turn anhydrous CuSO$_4$ blue or CoCl$_2$ pink & \multirow{4}{0.33\textwidth}{Hydrated salt, HCO$_3^-$ may be present} \\ \cline{3-4}
	
	& & Cracking sound with brown gas & NO$_3^-$ of Pb$^{2+}$ may be present \\ \cline{3-4}
	
	& & Cracking sound with no gas evolving & \multirow{2}{*}{Cl$^-$ of Na$^+$ may be present} \\ \cline{3-4}
	
	& & Residue yellow when hot and white when cold & \multirow{2}{*}{Zn$^{2+}$ may be present} \\ \cline{3-4}
	
	& & Residue reddish brown when hot and yellow when cold & \multirow{2}{*}{Pb$^{2+}$ may be present} \\ \cline{3-4}
	
	& & Black residue & Cu$^{2+}$ may be present \\ \cline{3-4} 
	
	& & Reddish brown residue & Fe$^{2+}$, Fe$^{3+}$ may be present \\ \cline{3-4}
	
	& & White residue & Ca$^{2+}$, Na$^+$ may be present \\ \hline
	
	% ACTION OF DILUTE HCl ACID ON A SOLID SAMPLE
	3. & \textbf{Action of Dilute HCl Acid on a Solid Sample} & \multirow{5}{0.31\textwidth}{Effervescence of a colourless gas evolves, which turns lime water milky and wet litmus paper from blue to red} & \multirow{5}{0.32\textwidth}{CO$_3^{2-}$, HCO$_3^-$ may be present} \\ 
	& Transfer about 0.5 g of solid sample in a test-tube followed by 3 drops of dilute HCl & & \\ \hline
	
	% ACTION OF CONCENTRATED H2SO4 ON A SOLID SAMPLE
	4. & \textbf{Action of Concentrated H$_2$SO$_4$ on a Solid Sample} & \multirow{6}{0.31\textwidth}{Effervescence of a colourless gas evolves, The gas turns lime water milky and wet litmus paper from blue to red} & \multirow{6}{0.31\textwidth}{CO$_3^{2-}$, HCO$_3^-$ may be present} \\
	
	& & & \\ 
	
	& \multirow{6}{0.3375\textwidth}{[\textit{\textbf{Safety Precautions:} Concentrated H$_2$SO$_4$ is corrosive. (a) Handle with care (b) Do not boil (c) Hold the test-tube in a slanting position and away from observers and neighbours}]} & & \\ 
	
	& & & \\
	
	& & &  \\ \cline{3-4}
	
	& & & \\
	
	& & Colourless gas evolves, which forms white dense fumes with ammonia & \multirow{3}{*}{Cl$^-$ may be present} \\
	\clearpage
	& Transfer about 0.5 g of a solid sample into a test-tube followed by 3 drops of concentrated H$_2$SO$_4$ acid. Dip a glass rod in concentrated ammonia solution and pass it to the mouth of a test tube containing the mixture. & \multirow{7}{0.31\textwidth}{Evolution of brown fumes, which turn wet litmus paper from blue to red and intensify on addition of copper turnings} & \multirow{7}{*}{NO$_3^-$ may be present}  \\ \cline{3-4}
	
	& & & \\ 
%	
%	& & \\ 
%	
%	& & & \\
%	
%	& & &  \\ \cline{3-4}
	
	& If no reaction warm the contents gently. Add copper turnings. Hold wet litmus paper on the mouth of the test-tube containing the mixture. & \multirow{4}{*}{No gas evolves} & \multirow{4}{*}{SO$_4^{2-}$ may be present} \\ \hline
	
	% FLAME TEST
	5. & \textbf{Flame Test} & & \\
	
	& & Golden yellow flame & Na$^+$ may be present \\ 
	
	& \multirow{6}{0.3375\textwidth}{\textit{Cleaning the test apparatus:} Dip a nichrome wire or glass rod or back side of the test-tube in concentrated HCl (in a watch glass) then heat it in a non-luminous flame.} & & \\ \cline{3-4}
	
	& & & \\

	& & Brick red flame & Ca$^{2+}$ may be present \\ 
	
	& & & \\ \cline{3-4}
	
	& & & \\ 
	
	& & Bluish-green flame & Cu$^{2+}$ may be present \\ 
	
	& & & \\ \cline{3-4}
	
	& \multirow{1}{0.3375\textwidth}{\textit{Test:} Dip the cleaned wire (or glass rod or test-tube) in concentrated HCl to allow a small sample to adhere on it. Pick up a small amount of the smaple using a wet wire (or glass rod or test-tube) and heat it on a flame} & & \\
	
	& & Bluish-white (pale-blue) flame & Pb$^{2+}$ may be present \\ 
	
	& & & \\ \cline{3-4}
	
	& & & \\
	
	& & Yellow (orange) sparks & Fe$^{2+}$, Fe$^{3+}$ may be present \\ 
	
	& & & \\ \cline{3-4}
	
	& & \multirow{2}{*}{No definite colour seen} & \multirow{2}{*}{Zn$^{2+}$, NH$_4^+$ may be present} \\
	
	& & & \\ \hline
	
	% SOLUBILITY OF SOLID SAMPLES
	6. & \textbf{Solubility of Solid Samples} & \multirow{6}{0.31\textwidth}{Soluble forming colourless solution} & Na$^+$, NH$_4^+$, NO$_3^-$ may be present \\ \cline{4-4}
	
	& \multirow{7}{0.3375\textwidth}{Transfer about 0.5 g of the solid sample into the test tube and add enough cold distilled water to dissolve the solid sample. If the sample does not dissolve warm the contents.} & & CO$_3^{2-}$, HCO$_3^-$, of Na$^+$, NH$_4^+$ may be present \\ \cline{4-4}
	
	& & & Cl$^-$ of Zn$^{2+}$ or Ca$^{2+}$ may be present \\ \cline{4-4}
	
	& & & SO$_4^{2-}$ of Zn$^{2+}$ may be present \\ \cline{3-4}
	
	& & Soluble forming blue solution & Cu$^{2+}$ may be present \\ \cline{3-4}
	
	& & Soluble forming pale green solution & \multirow{2}{*}{Fe$^{2+}$ may be present} \\ \cline{3-4}
	
	& & Soluble forming yellowish-brown solution & \multirow{2}{*}{Fe$^{3+}$ may be present} \\ \cline{3-4}
	
	& & Insoluble in cold water but soluble in hot water. Crystals reappear on cooling & \multirow{3}{*}{Cl$^-$ of Pb$^{2+}$ may be present} \\ \cline{3-4}
	
	& & \multirow{4}{*}{Insoluble} & CO$_3^{2-}$ of Ca$^{2+}$, Pb$^{2+}$, Zn$^{2+}$, Fe$^{2+}$, Fe$^{3+}$, Cu$^{2+}$ may be present \\ \cline{4-4}
	
	& & & SO$_4^{2-}$ of Ca$^{2+}$, Pb$^{2+}$ may be present \\
	\end{longtable}
\end{center}

%%%%%%%%%%%%%%%%%%%%%%%%%%%%%%%%%%%%%%%%%%%%%%%%%%
%--------------------------------------------------TESTS IN SOLUTION---------------------------------------------------
%%%%%%%%%%%%%%%%%%%%%%%%%%%%%%%%%%%%%%%%%%%%%%%%%%
\vspace*{-10mm}
\subsubsection{TESTS IN SOLUTION}
\noindent \textbf{Preparation of the Stock Solution of the Sample} \\
\noindent Transfer about 1 g of the solid sample in a test-tube. Add enough amount of distilled water (15-20 cm$^3$) and shake thoroughly. If the sample is insoluble in cold water, warm the contents. If the sample is insoluble in hot water, transfer about 1 g of the new solid sample in a test-tube, and then dissolve it in dilute nitric acid (to about 15-20 cm$^3$ of the final solution). Peform the tests shown in Table \ref{chem_test_sol}

\begin{center}
	\begin{longtable}{|p{0.0125\textwidth}p{0.3375\textwidth}|p{0.31\textwidth}|p{0.32\textwidth}|} 		
	
	\multicolumn{4}{c}
	{{\bfseries \tablename\ \thetable{} -- Tests in Solution}} \label{chem_test_sol} \\
	\hline \multicolumn{2}{|p{0.35\textwidth}|}{\textbf{Experiment to be Performed}} & \multicolumn{1}{|p{0.31\textwidth}|}{\textbf{Expected Observations}} & \multicolumn{1}{|p{0.32\textwidth}|}{\textbf{Inference}} \\ \hline
	\endfirsthead
		
	\multicolumn{4}{c}
	{{\bfseries \tablename\ \thetable{} -- continued from previous page}} \\
	\hline \multicolumn{2}{|c|}{\textbf{Experiment to be Performed}} & \multicolumn{1}{|c|}{\textbf{Expected Observations}} & \multicolumn{1}{|c|}{\textbf{Inference}} \\ \hline
	\endhead
	
	\hline \multicolumn{4}{r}{{Continued on next page}} \\
	\endfoot
	
	\hline
	\endlastfoot

	% ACTION OF NaOH SOLUTION ON A SAMPLE SOLUTION
	1. & \textbf{Action of NaOH Solution on a Sample Solution} & White precipitate is formed, soluble in excess & \multirow{2}{*}{Zn$^{2+}$, Pb$^{2+}$ may be present} \\ \cline{3-4}
	
	& \multirow{5}{0.3375\textwidth}{To about 1 cm$^3$ of the original sample solution, add sodium hydroxide solution drop-wise until in excess} & White precipitate is formed, insoluble in excess & \multirow{2}{*}{Ca$^{2+}$ may be present} \\ \cline{3-4}
	
	& & Blue precipitate is formed, insoluble in excess & \multirow{2}{*}{Cu$^{2+}$ may be present} \\ \cline{3-4}
	
	& & Green precipitate is formed, insoluble in excess, which turns brown on standing & \multirow{3}{*}{Fe$^{2+}$ may be present} \\ \cline{3-4}
	
	& & Reddish-brown precipitate is formed, which is insoluble in excess & \multirow{3}{*}{Fe$^{3+}$ may be present} \\ \cline{3-4}
	
	& & No precipitate is formed; on warming, a colourless gas with a pungent choking smell which turns wet litmus paper from red to blue evolves & \multirow{5}{*}{NH$_4^+$ may be present} \\ \cline{3-4}
	
	& & No precipitate is formed, even on warming & \multirow{2}{*}{Na$^+$ may be present} \\ \hline
	
	% ACTION OF NH3 SOLUTION ON A SAMPLE SOLUTION
	2. & \textbf{Action of NH$_3$ Solution on a Sample Solution} & White precipitate is formed, insoluble in excess & \multirow{2}{*}{Pb$^{2+}$ may be present} \\ \cline{3-4}
	
	& \multirow{2}{0.3375\textwidth}{To about 1 cm$^3$ of the original sample solution, add ammonia } & White gelatinous precipitate is formed, soluble in excess & \multirow{2}{*}{Zn$^{2+}$ may be present} \\ \cline{3-4}
	
	& solution drop-wise until in excess & No precipitate is formed & Ca$^{2+}$, Na$^+$, NH$_4^+$ may be present \\ \cline{3-4}
	
	& & Pale blue precipitate is formed, soluble in excess forming deep blue solution & \multirow{3}{*}{Cu$^{2+}$ may be present} \\ \cline{3-4} 
	
	& & Green precipitate is formed, insoluble in excess & \multirow{2}{*}{Fe$^{2+}$ may be present} \\ \cline{3-4}
	
	& & Reddish-brown precipitate is formed, insoluble in excess & \multirow{2}{*}{Fe$^{3+}$ may be present} \\ 
	\end{longtable}
\end{center}

%%%%%%%%%%%%%%%%%%%%%%%%%%%%%%%%%%%%%%%%%%%%%%%%%%
%--------------------------------------------CONFIRMATORY TESTS---------------------------------------------------
%%%%%%%%%%%%%%%%%%%%%%%%%%%%%%%%%%%%%%%%%%%%%%%%%%

%--------------------------------------------CATIONS---------------------------------------------------
\vspace*{-10mm}
\subsubsection{CONFIRMATORY TESTS}

\begin{center}
	\begin{longtable}{|p{0.0125\textwidth}p{0.3375\textwidth}|p{0.31\textwidth}|p{0.32\textwidth}|} 		
	
	\multicolumn{4}{c}
	{{\bfseries \tablename\ \thetable{} -- Confirmation Tests for Cations}} \label{chem_test_conf} \\
	\hline \multicolumn{2}{|p{0.35\textwidth}|}{\textbf{Experiment to be Performed}} & \multicolumn{1}{|p{0.31\textwidth}|}{\textbf{Expected Observations}} & \multicolumn{1}{|p{0.32\textwidth}|}{\textbf{Inference}} \\ \hline
	\endfirsthead
		
	\multicolumn{4}{c}
	{{\bfseries \tablename\ \thetable{} -- continued from previous page}} \\
	\hline \multicolumn{2}{|c|}{\textbf{Experiment to be Performed}} & \multicolumn{1}{|c|}{\textbf{Expected Observations}} & \multicolumn{1}{|c|}{\textbf{Inference}} \\ \hline
	\endhead
	
	\hline \multicolumn{4}{r}{{Continued on next page}} \\
	\endfoot
	
	\hline
	\endlastfoot

	% CONFIRMATORY TESTS FOR Ca2+
	1. & \textbf{Confirmatory Tests for Ca$^{2+}$} &  &  \\ 
	& (i) To about 1 cm$^3$ of the original sample solution, add excess ammonia solution followed by ammonium oxalate. & \multirow{4}{*}{White precipitate is formed} & \multirow{4}{*}{Ca$^{2+}$ confirmed} \\ 
	
	& & & \\
	
	& (ii) Perform flame test & Brick-red & Ca$^{2+}$ confirmed \\ \hline
	
	% CONFIRMATORY TESTS FOR Pb2+
	2. & \textbf{Confirmatory Tests for Pb$^{2+}$} & & \\
	& (i) To about 1 cm$^3$ of the sample solution, add K$_2$CrO$_4$ & \multirow{2}{*}{Yellow precipitate is formed} & \multirow{2}{*}{Pb$^{2+}$ confirmed} \\ 
	
	& & & \\
	
	& (ii) To about 1 cm$^3$ of the sample solution, add KI solution. Warm and cool the mixture. & Yellow precipitate which disappears on warming but re-appears on cooling & \multirow{3}{*}{Pb$^{2+}$ confirmed} \\ \hline
	
	% CONFIRMATORY TESTS FOR Zn2+
	3. & \textbf{Confirmatory Tests for Zn$^{2+}$} & & \\
	& (i) To about 1 cm$^3$ of the sample solution, add potassium hexacyanoferrate (II) solution followed by few drops of dilute HCl. & \multirow{4}{0.31\textwidth}{Bluish white precipitate insoluble in dilute HCl} & \multirow{4}{*}{Zn$^{2+}$  confirmed} \\
	
	& & & \\ 
	
	& (ii) To about 1 cm$^3$ of the sample solution, add dilute NaOH solution until excess & \multirow{3}{0.31\textwidth}{White precipitate soluble in excess} & \multirow{3}{*}{Zn$^{2+}$ confirmed} \\ \hline
	
	% CONFIRMATORY TESTS FOR NH4+
	4. & \textbf{Confirmatory Tests for NH$_4^+$} & & \\
	& (i) To about 1 cm$^3$ of the original sample solution, add about 2-3 drops of Nessler's reagent. & \multirow{3}{0.31\textwidth}{Reddish-brown precipitate is formed} & \multirow{3}{*}{NH$_4^+$ confirmed} \\
	
	& & & \\
	
	& (ii) Transfer about 0.2 g of the original solid sample in a test-tube, add sodium hydroxide solution just to cover the whole solid then warm gently. Test for gas evolved. & \multirow{5}{0.31\textwidth}{Colourless gas evolves which turns wet litmus paper from red to blue} & \multirow{5}{*}{NH$_4^+$ confirmed} \\ \hline
	
	% CONFIRMATORY TEST FOR Na+
	5. & \textbf{Confirmatory Test for Na$^+$} & & \\
	& Perform flame test & Yellow flame & Na$^+$ confirmed \\ \hline
	
	% CONFIRMATORY TESTS FOR Cu2+
	6. & \textbf{Confirmatory Tests for Cu$^{2+}$} & & \\
	& (i) To about 1 cm$^3$ of the original sample solution, add ammonia solution drop-wise until in excess. & Pale blue precipitate soluble in excess of aqueous ammonia forming deep blue solution. & \multirow{3}{*}{Cu$^{2+}$ confirmed} \\
	
	& & & \\
	
	& (ii) To about 1 cm$^3$ of the original sample solution, add potassium hexacyanoferrate (II) & \multirow{3}{0.31\textwidth}{Reddish-brown precipitate} & \multirow{3}{*}{Cu$^{2+}$ confirmed} \\ \hline
	
	% CONFIRMATORY TESTS FOR Fe2+, Fe3+
	7. & \textbf{Confirmatory Tests for Fe$^{2+}$, Fe$^{3+}$} & & \\
	& (i) To about 1 cm$^3$ of the sample solution, add few drops of potassium hexacyanoferrate (III). & \multirow{3}{*}{Deep blue precipitate} & \multirow{3}{*}{Fe$^{2+}$ confirmed} \\ 
	
	& & & \\
	
	& (ii) To about 1 cm$^3$ of the sample solution, add few drops of potassium hexacyanoferrate (II). & \multirow{3}{*}{Deep blue precipitate} & \multirow{3}{*}{Fe$^{3+}$ confirmed} \\ 
	
	& & & \\
	
	& (iii) To about 1 cm$^3$ of the sample solution, add few drops of potassium or ammonium thiocyanate solution. & \multirow{4}{*}{Deep blood-red solution} & \multirow{4}{*}{Fe$^{3+}$ confirmed} \\ 
	\end{longtable}
\end{center}

\clearpage
%--------------------------------------------ANIONS---------------------------------------------------
\vspace*{-10mm}
\begin{center}
	\begin{longtable}{|p{0.0125\textwidth}p{0.3375\textwidth}|p{0.31\textwidth}|p{0.32\textwidth}|} 		
	
	\multicolumn{4}{c}
	{{\bfseries \tablename\ \thetable{} -- Confirmation Tests for Anions}} \label{chem_test_conf} \\
	\hline \multicolumn{2}{|p{0.35\textwidth}|}{\textbf{Experiment to be Performed}} & \multicolumn{1}{|p{0.31\textwidth}|}{\textbf{Expected Observations}} & \multicolumn{1}{|p{0.32\textwidth}|}{\textbf{Inference}} \\ \hline
	\endfirsthead
		
	\multicolumn{4}{c}
	{{\bfseries \tablename\ \thetable{} -- continued from previous page}} \\
	\hline \multicolumn{2}{|c|}{\textbf{Experiment to be Performed}} & \multicolumn{1}{|c|}{\textbf{Expected Observations}} & \multicolumn{1}{|c|}{\textbf{Inference}} \\ \hline
	\endhead
	
	\hline \multicolumn{4}{r}{{Continued on next page}} \\
	\endfoot
	
	\hline
	\endlastfoot

	% CONFIRMATORY TESTS FOR SO4(2-)
	1. & \textbf{Confirmatory Tests for SO$_4^{2-}$} & & \\
	& (i) Transfer about 1 cm$^3$ of the original sample solution into the test-tube. Add barium chloride followed by dilute HCl or barium nitrate followed by dilute HNO$_3$. & \multirow{5}{*}{White precipitate} & \multirow{5}{*}{SO$_4^{2-}$ confirmed} \\
	
	& & & \\
	
	& (ii) Transfer about 1 cm$^3$ of the original sample solution into the test-tube. Add ethanoic acid followed by lead ethanoate. Divide the resulting mixture into two portions. In one portion add dilute HCl and in another add ammonium ethanoate solution. & \multirow{8}{0.31\textwidth}{White precipitate insoluble in dilute HCl but soluble in ammonium ethanoate solution} & \multirow{8}{*}{SO$_4^{2-}$ confirmed} \\ \hline
	
	% CONFIRMATORY TESTS FOR NO3-
	2. & \textbf{Confirmatory Tests for NO$_3^-$} & & \\
	& (i) Transfer about 1 cm$^3$ of the original sample solution into the test-tube. Add dilute H$_2$SO$_4$; then add freshly prepared FeSO$_4$ solution followed by \textbf{careful} addition of concentrated H$_2$SO$_4$ along the side of the test-tube. & \multirow{7}{0.31\textwidth}{Brown ring is formed at the junction of the liquids} & \multirow{7}{*}{NO$_3^-$ confirmed} \\
	
	& & & \\
	
	& (ii) Transfer about 0.5 g of the original solid sample into the test-tube. Add copper turnings followed by concentrated H$_2$SO$_4$ then warm. & \multirow{4}{*}{Brown fumes evolve} & \multirow{4}{*}{NO$_3^-$ confirmed} \\ \hline
	
	% CONFIRMATORY TESTS FOR CO3(2-), HCO3-
	3. & \textbf{Confirmatory Tests for CO$_3^{2-}$, HCO$_3^-$} & & \\
	& (i) Transfer about 1 cm$^3$ of the original sample solution into a test-tube. Add few drops of MgSO$_4$ solution. If no precipitate is formed, warm the contents. & \multirow{5}{0.31\textwidth}{White precipitate is formed before warming the contents} & \multirow{5}{0.31\textwidth}{CO$_3^{2-}$ confirmed} \\
	
	& & White precipitate is formed after warming the contents & \multirow{2}{*}{HCO$_3^-$ confirmed} \\ 
	
	& & & \\
	
	& (ii) Transfer about 1 cm$^3$ of the original sample solution into a test-tube. Add BaCl$_2$ solution. If the precipitate forms, add dilute HCl. & \multirow{4}{0.31\textwidth}{White precipitate soluble in dilute HCl is formed} & \multirow{4}{*}{CO$_3^{2-}$ confirmed} \\
	
	& & & \\
	
	& (iii) Transfer about 0.5 g of water-insoluble solid sample in a test-tube. Add about 1 cm$^3$ of dilute nitric acid. & \multirow{4}{0.31\textwidth}{Effervescence of a colourless gas, which turns lime water milky} & \multirow{4}{*}{CO$_3^{2-}$ confirmed} \\ \hline
	
	% CONFIRMATORY TEST FOR Cl-
	4. & \textbf{Confirmatory Test for Cl$^-$} & & \\ 
	
	& To about 1 cm$^3$ of the original sample solution, add about 3 drops of dilute nitric acid followed by about 3 drops of silver nitrate solution & \multirow{4}{*}{White precipitate is formed} & \multirow{4}{*}{Cl$^-$ confirmed} \\ 
	\end{longtable}
\end{center}


%----------------------------------------------------------------------------------------------------------------------------
\subsection{Tips and Tricks}

\noindent The following are some tips and tricks for successfully performing the Qualitative Analysis practical: 
\begin{itemize}[topsep=0ex,itemsep=0ex,partopsep=1ex,parsep=1ex]
	\item Make sure you use a qualitative analysis guide sheet; if your school does not have some request them
	\item Do all the steps
	\item Do two confirmation steps; one for the cation you think you have and one for the anion you think you have
\end{itemize}
%----------------------------------------------------------------------------------------------------------------------------

\subsection{Sample Practical Question}
%The following is a sample practical question from 2012.\\[6pt]
%
%
%Substance \textbf{V} is a simple salt which contains one cation and one anion. Carry our the experiments described below. Record carefully your observations and make appropriate inferences and hence identify the anion and cation present in sample \textbf{V}.\\
%
%\begin{center}
%\begin{tabular}{|l|p{8cm}|l|l|}
%\hline
%\textbf{S/n}&\textbf{Experiment}&\textbf{Observation}&\textbf{Inference}\\ \hline
%1&Observe the appearance of sample \textbf{V}.&&\\ \hline
%2&Put a little amount of sample \textbf{V} in a test tube then add water and shake.&&\\ \hline
%3&Heat a little amount of \textbf{V} in a dry test tube.&&\\ \hline
%{\multirow{4}{*}{4}}&To a little sample \textbf{V} in a test tube add dilute Hydrochloric acid. Add more of the acid until the test tube is half full. Divide the resulting solution into three portions and add the following:&&\\ \cline{2-4}
%&\begin{enumerate}
%\item[a)] To the one portion add NaOH solution drop wise then excess.
%\end{enumerate}&&\\ \cline{2-4}
%&\begin{enumerate}
%\item[b)] To the second portion add ammonia solution drop wise then in excess.
%\end{enumerate}&&\\ \cline{2-4}
%&\begin{enumerate}
%\item[c)] To the third portion add ammonium oxalate solution.
%\end{enumerate}&&\\ \hline
%5&Perform flame test.&&\\ \hline
%\end{tabular}\\
%
%\end{center}
%\newpage
%Conclusion\\
%\begin{enumerate}
%\item[(i)] The cation in sample \textbf{V} is \_\_\_\_ .
%\item[(ii)] The anion in sample \textbf{V} is \_\_\_\_ .
%\item[(iii)] The chemical formula of \textbf{V} is \_\_\_\_ .
%\item[(iv)] The name of compound \textbf{V} is \_\_\_\_ . \hfill \textbf{(20 marks)} \\
%\end{enumerate}

\noindent The following are two sample practical questions: 

\subsubsection{Sample Practical \#1}

\noindent Sample H contains one cation and one anion. Using systematic qualitative analysis procedures record carefully your experiments, observations and inferences as Table \ref{table_qual_sample_1} shows. Finally, identify the anion and cation present in sample H. 

\vspace*{-5mm}
\begin{center}
	\begin{longtable}{|c|p{0.25\textwidth}|p{0.25\textwidth}|p{0.25\textwidth}|} 		
	
	\multicolumn{4}{c}
	{{\bfseries \tablename\ \thetable{}}} \label{table_qual_sample_1} \\ \hline
	\textbf{S/n} & \multicolumn{1}{|c|}{\textbf{Experiment}} & \multicolumn{1}{|c|}{\textbf{Observation}} & \multicolumn{1}{|c|}{\textbf{Inference}} \\ \hline
	\endfirsthead
	
	\hline
	\endlastfoot
	
	& & & \\
	& & & \\
	& & & \\
	
	\end{longtable}
\end{center}

\vspace*{-7mm}
\noindent \textbf{Conclusion:}
\begin{enumerate}[topsep=0ex,itemsep=0ex,partopsep=1ex,parsep=1ex]
	\item[i)] The cation in sample H is:
	\item[ii)] The anion in sample H is:
\end{enumerate}


\subsubsection{Sample Practical \#2}

\noindent Sample Q is a simple salt containing one cation and one anion. Carefully carry out all the experiments described in the Table \ref{table_qual_sample_2}. Record all your observations and make appropriate inferences to identify the ions present in sample Q. 

\vspace*{-5mm}
\begin{center}
	\begin{longtable}{|c|cp{0.45\textwidth}|p{0.15\textwidth}|p{0.15\textwidth}|} 		
	
	\multicolumn{5}{c}
	{{\bfseries \tablename\ \thetable{}}} \label{table_qual_sample_2} \\ \hline
	\textbf{S/n} & \multicolumn{2}{|c|}{\textbf{Experiments}} & \multicolumn{1}{|c|}{\textbf{Observation}} & \multicolumn{1}{|c|}{\textbf{Inference}} \\ \hline
	\endfirsthead
	
	\hline
	\endlastfoot
	
	(a) & \multicolumn{2}{l|}{Observe sample Q} & & \\ \hline
	(b) & \multicolumn{2}{l|}{\multirow{1}{0.45\textwidth}{Put a spatulaful of sample Q in a test tube and add distilled water}} & & \\ 
	& & & & \\ \hline
	(c) & \multicolumn{2}{l|}{\multirow{1}{0.45\textwidth}{Transfer 0.5 g of sample Q in a test tube, add dilute HCl}} & & \\ 
	& & & & \\ \hline
	(d) & \multicolumn{2}{l|}{\multirow{1}{0.45\textwidth}{Transfer 0.5 g of sample Q in a test tube, add concentrated sulphuric acid}} & & \\ 
	& & & & \\ \hline
	(e) & \multicolumn{2}{l|}{\multirow{1}{0.45\textwidth}{Dissolve sample Q then divide the resulting solution into three portions.}} & & \\
	& & & & \\ 
	& (i) & To the first portion add sodium hydroxide solution & & \\ \cline{2-5}
	& (ii) & To the second portion add ammonia solution & & \\ \cline{2-5}
	& (iii) & To the third portion add potassium ferricyanide solution [K$_3$Fe(CN)$_6$] & & \\ \cline{2-5}
	& (iv) & To the fourth portion add lead acetate solution followed by acetic acid solution & & \\
	\end{longtable}
\end{center}

\vspace*{-7mm}
\noindent \textbf{Conclusion:} 
\begin{enumerate}[topsep=0ex,itemsep=0ex,partopsep=1ex,parsep=1ex]
	\item[(a)] 
	\begin{enumerate}[topsep=0ex,itemsep=0ex,partopsep=1ex,parsep=1ex]
		\item[i)] The cation in sample Q is \rule{1.5cm}{0.15mm}.
		\item[ii)] The anion in sample Q is \rule{1.5cm}{0.15mm}.
		\item[iii)] The compound Q is \rule{1.5cm}{0.15mm}.
	\end{enumerate}
	\item[(b)] Write the reaction equation that took place at experiment (c).
	\item[(c)] State three chemical properties of the metal in Q.
	\item[(d)] State two uses of Q. 
\end{enumerate}

%==============================================================================
%==============================================================================

\section{Chemical Kinetics and Equilibrium} \index{Practicals! Chemistry! chemical kinetics and equilibrium} \index{Chemical kinetics} \index{Equilibrium|see{Chemical kinetics}}
%[brief 1 paragraph explanation of practical]\\

This section contains the following:
\begin{itemize}[topsep=0ex,itemsep=0ex,partopsep=1ex,parsep=1ex]
	\item Chemical Kinetics and Equilibrium Theory
	\item Tips and Tricks
	\item Sample Practical Questions
\end{itemize}
%----------------------------------------------------------------------------------------------------------------------------
\subsection{Theory}
Compared to the other two NECTA chemistry practicals - Acid/Base Titration (i.e. Volumetric Analysis) and Qualitative Analysis - Chemical Kinetics has few alternative chemicals that can be used.\\

The chemical reaction in the NECTA exam is a precipitation of sulphur. 

\[ \mathrm{Na_2S_2O_3}_{(aq)} + \mathrm{2HCl}_{(aq)} \longrightarrow \mathrm{2NaCl}_{(aq)} + \mathrm{H}_{2}\mathrm{O}_{(l)} + \mathrm{SO_2}_{(g)} + \mathrm{S}_{(s)} \]

The preparation and procedure for Chemical Kinetics is very simple.

%----------------------------------------------------------------------------------------------------------------------------
\subsection{Tips and Tricks}

\noindent The following are some tips and tricks for successfully performing the Chemical Kinetics and Equilibrium practical: 
\begin{itemize}[topsep=0ex,itemsep=0ex,partopsep=1ex,parsep=1ex]
	\item If you have a low concentration the test may take over 4 minutes
	\item Normally the NECTA will ask for the reaction that goes from opaque to translucent, but it could be the other way around
	\item Make sure you start the stop watch IMMEDIATELY after combining the reactants
	\item Most common question is how concentration affects the rate of reaction
\end{itemize}

%----------------------------------------------------------------------------------------------------------------------------
\subsection{Sample Practical Question}
%The following is a sample practical question from 2012.\\[6pt]
%
%Your are provided with the following materials:\\
%\begin{enumerate}
%\item[ ] \textbf{ZO}:  A solution of 0.13 M Na$_2$S$_2$O$_3$ (sodium thiosulphate);
%\item[ ] \textbf{UU}:  A solution of 2 M HCl;
%\item[ ] Thermometer;
%\item[ ] Heat source/burner;
%\item[ ] Stopwatch.\\
%\end{enumerate}
%
%Procedure:\\
%\begin{enumerate}
%\item[(i)] Place 500 cm$^3$ beaker, which is half-filled with water, on the heat source as a water bath.
%\item[(ii)] Measure 10 cm$^3$ of \textbf{ZO} and 10 cm$^3$ of \textbf{UU} into two separate test tubes.
%\item[(iii)] Put the two test tubes containing \textbf{ZO} and \textbf{UU} solutions into a water bath.
%\item[(iv)] When the solutions attain a temperature of 60$^o$C, remove the test tubes from the water bath and pour both solutions into 100 cm$^3$ empty beaker and immediately start the stop watch.
%\item[(v)] Place the beaker with the contents on top of a piece of paper marked \textbf{X}.
%\item[(vi)] Note the time taken for the mark \textbf{X} to disappear.
%\item[(vii)] Repeat step (i) to (vi) at temperature 70$^o$C, 80$^o$C and 90$^o$C.
%\item[(viii)] Record your results as in Table 1.
%\end{enumerate}
%
%\begin{center}
%\begin{tabular}{|p{5cm}|p{5cm}|p{3cm}|}
%\multicolumn{1}{l}{Table 1}&\multicolumn{1}{l}{ }&\multicolumn{1}{l}{ }\\ \hline
%\textbf{Experiment}&\textbf{Temperature}&\textbf{Time (s)}\\ \hline
%1&60$^o$C&\\ \hline
%2&70$^o$C&\\ \hline
%3&80$^o$C&\\ \hline
%4&90$^o$C&\\ \hline
%\end{tabular}
%\end{center}
%
%\textbf{Questions:}\\
%\begin{enumerate}
%\item Write a balanced chemical equation for reaction between \textbf{UU} and \textbf{ZO}.
%\item What is the product which causes the solution to cloud the letter \textbf{X}?
%\item Plot a graph of temperature against time (s).
%\item What conclusion can you draw from you graph? \hfill \textbf{(20 marks)} \\
%\end{enumerate}
%
%
%\subsubsection{Discussion}
%This particular example was to investigate how temperature affects the rate of a chemical reaction.\\
%Other experiments for chemical kinetics involve concentration, surface area, and a catalyst; however, the most common problem statement is how concentration affects the rate of reaction.\\
%For all scenarios, make sure to tell students to start the stopwatch \textit{immediately} after combining the reactants.\\

\noindent The following are two sample practical questions:

\subsubsection{Sample Practical \# 1}

\noindent You are provided with the following: \\
\noindent \textbf{Solution M:} 0.2 M sodium thiosulphate (Na$_2$S$_2$O$_3$); \\
\noindent \textbf{Solution N:} 2 M hydrochloric acid (HCl); \\
\noindent Distilled water labeled W; \\
\noindent A sheet of water paper marked X; \\
\noindent Stop watch. \\

\noindent \textbf{Procedure} \\
\begin{enumerate}[topsep=0ex,itemsep=0ex,partopsep=1ex,parsep=1ex]
	\item[i)] Put a small beaker (100 cm$^3$) on top of the mark X on a sheet of paper in such a way that the mark is clearly seen through the top of the beaker. 
	\item[ii)] Measure 50 cm$^3$ of solution M and pour into a small beaker.
	\item[iii)] Using different measuring cylinder measure 10 cm$^3$ of solution N.
	\item[iv)] Start a stop watch simultaneously as you pour solution N in the beaker containing solution M.
	\item[v)] Stir the mixture with glass rod until the cross disappears
	\item[vi)] Stop the watch when the cross is out of sight. Record the time taken.
	\item[vii)] Repeat the whole process using 40 cm$^3$, 30 cm$^3$, 20 cm$^3$ and finally 10 cm$^3$ of solution M as shown in the Table \ref{table_kinetics_sample_1}. Top up solution M with W to make 50 cm$^3$ in each experiment before adding solution N. 
\end{enumerate}

\vspace*{-5mm}
\begin{center}
	\begin{longtable}{|c|c|c|c|c|c|} 		
	
	\multicolumn{6}{c}
	{{\bfseries \tablename\ \thetable{}}} \label{table_kinetics_sample_1} \\ \hline
	\textbf{Volume} & \textbf{Volume} & \textbf{Volume} & \textbf{Conc. of M after} & \textbf{Time for cross} & \textbf{Rate} \\
	\textbf{of M} & \textbf{of water} & \textbf{of N} & \textbf{adding water} & \textbf{to disappear} & \textbf{(sec$^{-1}$)} \\
	\textbf{(cm$^3$)} & \textbf{(cm$^3$)} & \textbf{(cm$^3$)} & \textbf{(moldm$^{-3}$)} & \textbf{(sec)} & \\ \hline
	\endfirsthead
	
	\hline
	\endlastfoot
	
	50 & 00 & 10 & & & \\ \hline
	40 & 10 & 10 & & & \\ \hline
	30 & 20 & 10 & & & \\ \hline
	20 & 30 & 10 & & & \\ \hline
	10 & 40 & 10 & & & \\
	\end{longtable}
\end{center}

\vspace*{-7mm}
\noindent \textbf{Questions}
\begin{enumerate}[topsep=0ex,itemsep=0ex,partopsep=1ex,parsep=1ex]
	\item[(a)] Complete filling the Table \ref{table_kinetics_sample_1}
	\item[(b)] Write down a balanced chemical equation for the reaction between M and N.
	\item[(c)] What substance was produced during the reaction which obscured the cross?
	\item[(d)] Use the data in Table \ref{table_kinetics_sample_1} to draw the following graphs:
	\begin{enumerate}[topsep=0ex,itemsep=0ex,partopsep=1ex,parsep=1ex]
		\item[i)] Concentration-time graph; concentration on the y-axis and time on the x-axis
		\item[ii)] Concentration-rate graph; concentration on the y-axis and rate on the x-axis
	\end{enumerate}
	\item[(e)] What conclusion can you draw from the results of the experiment?
\end{enumerate}

\subsubsection{Sample Practical \# 2}

\noindent You are provided with the following: \\
\noindent \textbf{E:} A solution made by dissolving 20 g of sodium thiosulphate in 1 dm$^3$ of the solution; \\
\noindent \textbf{F:} A solution of 2 M nitric acid; \\
\noindent \textbf{G:} Distilled water; \\
\noindent A sheet of white paper; \\
\noindent Thermometer; \\
\noindent Stop watch. \\

\noindent \textbf{Procedure} \\
\begin{enumerate}[topsep=0ex,itemsep=0ex,partopsep=1ex,parsep=1ex]
	\item[i)] Put a beaker (100 cm$^3$) on top of the cross drawn on the given sheet of paper. 
	\item[ii)] Measure 25 cm$^3$ of E using a measuring cylinder and pour it into the beaker in i)
	\item[iii)] Using another measuring cylinder measure 5 cm$^3$ of F and pour it into a beaker containing E and instantly start a stop watch.
	\item[iv)] Stir the mixture with a glass rod while you keep on observing the cross from above; record the time taken for the cross to disappear
	\item[v)] Repeat the procedures for different concentrations of E by taking 20 cm$^3$, 15 cm$^3$, 10 cm$^3$, and 5 cm$^3$ of the original E and making the total volume up to 25 cm$^3$ by adding G. Record the results as shown in the Table \ref{table_kinetics_sample_2}
\end{enumerate}

\vspace*{-5mm}
\begin{center}
	\begin{longtable}{|c|c|c|}	
	
	\multicolumn{3}{c}
	{{\bfseries \tablename\ \thetable{}}} \label{table_kinetics_sample_2} \\ \hline
	\textbf{Volume of E} & \textbf{Volume of G} & \textbf{Time taken for the cross to disappear} \\
	\textbf{(cm$^3$)} & \textbf{(cm$^3$)} & \textbf{(sec)} \\ \hline
	\endfirsthead
	
	\hline
	\endlastfoot
	
	25 & & \\ \hline
	20 & & \\ \hline
	15 & & \\ \hline
	10 & & \\ \hline
	5 & & \\
	\end{longtable}
\end{center}

\noindent \textbf{Questions}
\begin{enumerate}[topsep=0ex,itemsep=0ex,partopsep=1ex,parsep=1ex]
	\item[(a)] Complete filling the Table \ref{table_kinetics_sample_2}
	\item[(b)] 
	\begin{enumerate}[topsep=0ex,itemsep=0ex,partopsep=1ex,parsep=1ex]
		\item[i)] Using the data in the table, plot a volume-time graph (volume on the y-axis and the time in seconds on the x-axis)
		\item[ii)] What does the shape of the graph indicate?
	\end{enumerate}
	\item[(c)] Write down the ionic equation of the reaction between E and F
	\item[(d)] Why did the cross disappear?
	\item[(e)] Write two uses of the product which obscured the cross.
\end{enumerate}







